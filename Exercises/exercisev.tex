\documentclass[12pt, a4paper]{article}

\pdfoutput=1

\usepackage{exsheets}

\usepackage{mathrsfs, amsmath, amsthm, amssymb, stmaryrd, enumerate}%
\usepackage{tikz}%
\usepackage[linktocpage]{hyperref}%
%\usepackage{cleveref}
\usepackage[all]{xy}%
\xyoption{2cell}%
\UseAllTwocells%

\newdir{t>}{{\UseTips\dir{>}}}

%\creflabelformat{enumi}{(#2#1#3)}

%\hypersetup{
% pdfauthor={Daniel Schaeppi},
% pdfkeywords={Pushouts} {algebraic stacks} {Quasi-coherent sheaves} {weakly Tannakian categories}
%}


%********************************* MACROS ************************************%

\DeclareMathOperator{\Ob}{Ob}
\DeclareMathOperator{\id}{id}
\DeclareMathOperator{\el}{el}
\DeclareMathOperator{\op}{op}
\DeclareMathOperator{\Lan}{Lan}
\DeclareMathOperator{\Ran}{Ran}
\DeclareMathOperator{\Mod}{\mathbf{Mod}}
\DeclareMathOperator{\Rep}{Rep}
\DeclareMathOperator{\Vect}{\mathbf{Vect}}
\DeclareMathOperator{\Coalg}{\mathbf{Coalg}}
\DeclareMathOperator{\fgp}{fgp}
\DeclareMathOperator{\fp}{fp}
\DeclareMathOperator{\fd}{fd}
\DeclareMathOperator{\Hom}{Hom}
\DeclareMathOperator{\End}{End}
\DeclareMathOperator{\Spec}{Spec}

\DeclareMathOperator{\Tor}{Tor}
\DeclareMathOperator{\reg}{reg}
\DeclareMathOperator{\flt}{flat}


\DeclareMathOperator{\rk}{rk}
\DeclareMathOperator{\Sym}{Sym}
\DeclareMathOperator{\sgn}{sgn}
\DeclareMathOperator{\coev}{coev}
\DeclareMathOperator{\ev}{ev}

\DeclareMathOperator{\fl}{fin. \ell}

\DeclareMathOperator{\fin}{fin}
\DeclareMathOperator{\fg}{fg}
\DeclareMathOperator{\idem}{idem}
\DeclareMathOperator{\Modtwo}{\mathcal{M}}

\DeclareMathOperator{\iso}{iso}

\DeclareMathOperator{\LFdi}{\LF^{\rk d}_{\iso}}

\DeclareMathOperator{\dom}{dom}
\DeclareMathOperator{\Sh}{Sh}
\DeclareMathOperator{\Rex}{\mathbf{Rex}}

\DeclareMathOperator{\SymMonCat}{SymMonCat}


\DeclareMathOperator{\PsCone}{PsCone}


\DeclareMathOperator{\Ind}{Ind}
\DeclareMathOperator{\Span}{\mathbf{Span}}
\DeclareMathOperator{\Cospan}{\mathbf{Cospan}}
\DeclareMathOperator{\EM}{EM}

\DeclareMathOperator{\Aff}{\mathbf{Aff}}
\DeclareMathOperator{\Alg}{\mathbf{Alg}}
\DeclareMathOperator{\Coh}{\mathbf{Coh}}
\DeclareMathOperator{\QCoh}{\mathbf{QCoh}}
\DeclareMathOperator{\VB}{\mathbf{VB}}

\DeclareMathOperator{\BGL}{\mathrm{BGL}}

\DeclareMathOperator{\Cov}{Cov}
\DeclareMathOperator{\fpqc}{\mathit{fpqc}}

\DeclareMathOperator{\Aut}{Aut}
\DeclareMathOperator{\lax}{lax}
\DeclareMathOperator{\Tors}{Tors}


\DeclareMathOperator{\Map}{Map}
\DeclareMathOperator{\target}{target}
\DeclareMathOperator{\cospan}{Cospan}
\DeclareMathOperator{\CommAlg}{\mathbf{CommAlg}}
\DeclareMathOperator{\coop}{coop}
\DeclareMathOperator{\Coact}{Coact}
\DeclareMathOperator{\Gray}{\mathbf{Gray}}
\DeclareMathOperator{\PsMon}{\mathbf{PsMon}}
\DeclareMathOperator{\BrPsMon}{\mathbf{BrPsMon}}
\DeclareMathOperator{\SymPsMon}{\mathbf{SymPsMon}}
\DeclareMathOperator{\MonComon}{\mathbf{MonComon}}
\DeclareMathOperator{\BrMonComon}{\mathbf{BrMonComon}}
\DeclareMathOperator{\SymMonComon}{\mathbf{SymMonComon}}
\DeclareMathOperator{\Psa}{\mathbf{Psa}}

\DeclareMathOperator{\pr}{pr}

\DeclareMathOperator{\len}{\ell}
\DeclareMathOperator{\proj}{proj}
\DeclareMathOperator{\Fil}{Fil}
\DeclareMathOperator{\MF}{MF}
\DeclareMathOperator{\colim}{colim}
\DeclareMathOperator{\Er}{Er}

\DeclareMathOperator{\Ps}{\mathbf{Ps}}

\DeclareMathOperator{\Cocts}{\mathbf{Cocts}}
\DeclareMathOperator{\Lex}{\mathbf{Lex}}
\DeclareMathOperator{\cat}{\mathbf{cat}}
\DeclareMathOperator{\Cat}{\mathbf{Cat}}
\DeclareMathOperator{\Bicat}{\mathbf{Bicat}}
\DeclareMathOperator{\Tricat}{\mathbf{Tricat}}
\DeclareMathOperator{\Gpd}{\mathbf{Gpd}}
\DeclareMathOperator{\Mky}{\mathbf{Mky}}
\DeclareMathOperator{\CAT}{\mathbf{CAT}}
\DeclareMathOperator{\Set}{\mathbf{Set}}
\DeclareMathOperator{\Ab}{\mathbf{Ab}}
\DeclareMathOperator{\CGTop}{\mathbf{CGTop}}
\DeclareMathOperator{\Mon}{\mathbf{Mon}}
\DeclareMathOperator{\Comon}{\mathbf{Comon}}
\DeclareMathOperator{\Comod}{\mathbf{Comod}}

\DeclareMathOperator{\Nat}{\mathrm{Nat}}
\DeclareMathOperator{\Fun}{\mathrm{Fun}}
\DeclareMathOperator{\LF}{\mathrm{LF}}

\DeclareMathOperator{\CAlg}{\mathrm{CAlg}}

\newcommand{\ca}[1]{\mathscr{#1}}
\newcommand{\VNat}{\ca{V}\mbox{-}\Nat}
\newcommand{\Vcat}{\ca{V}\mbox{-}\Cat}
\newcommand{\VCAT}{\ca{V}\mbox{-}\CAT}
\newcommand{\VCATlp}{\ca{V}\mbox{-}\CAT_{\mathrm{lp}}}
\newcommand{\VCATc}{\ca{V}^{\prime}\mbox{-}\Cat_{\mathrm{c}}}

\newcommand{\Catfc}{\Cat_{\mathrm{fc}}}
\newcommand{\VCatfc}{\ca{V}\mbox{-}\Cat_{\mathrm{fc}}}
\newcommand{\kten}{\mathop{\boxtimes_{\mathrm{fc}}}}

\newcommand{\Prs}[1]{\mathcal{P}\ca{#1}}
\newcommand{\Bimod}[1]{{_\ca{#1}}{\mathcal{M}}{_\ca{#1}}}
\newcommand{\modules}[2]{{_\ca{#1}}{\mathcal{M}}{_\ca{#2}}}
\newcommand{\CC}[1]{\mathbf{Comon}\left(\Bimod{#1}\right)}

\DeclareMathOperator{\U}{O}

\newcommand{\ubar}[1]{\underline{#1\mkern-4mu}\mkern4mu }

\newcommand{\dual}[1]{{#1}^{\circ}}
\newcommand{\ldual}[1]{{#1}^{\vee}}


\newcommand{\ten}[1]{\mathop{{\otimes}_{#1}}}
\newcommand{\tenl}[1]{\mathop{{}_{#1}{\otimes}}}
\newcommand{\tenlr}[2]{\mathop{{}_{#1}{\otimes}_{#2}}}

\newcommand{\boxten}[1]{\mathop{{\boxtimes}_{#1}}}

\newcommand{\pb}[1]{\mathop{{\times}_{#1}}}
\newcommand{\po}[1]{\mathop{{+}_{#1}}}


\newcommand{\defl}{\mathrel{\mathop:}=}


% THEOREM ENVIRONMENTS

\theoremstyle{plain}
\newtheorem{thm}{Theorem}[subsection]
\newtheorem*{thm*}{Theorem}
\newtheorem{prop}[thm]{Proposition}
\newtheorem{lemma}[thm]{Lemma}
\newtheorem{cor}[thm]{Corollary}

\theoremstyle{definition}
\newtheorem{example}[thm]{Example}
\newtheorem{rmk}[thm]{Remark}
\newtheorem{dfn}[thm]{Definition}
\newtheorem{notation}[thm]{Notation}

\newtheoremstyle{citing}{}{}{\itshape}{}{\bfseries}{.}{ }{\thmnote{#3}}
\theoremstyle{citing}
\newtheorem{cit}{}

\newtheoremstyle{citingdfn}{}{}{}{}{\bfseries}{.}{ }{\thmnote{#3}}
\theoremstyle{citingdfn}
\newtheorem{citdfn}{}


\numberwithin{equation}{section}

%\keywords{Fiber functors, Tannakian categories}
%\subjclass[2010]{14A20, 18D10}

%\author{Daniel Sch\"appi}
%\thanks{This research was supported by the DFG grant: SFB 1085 ``Higher invariants.''}
%\address{Fakult{\"a}t f{\"u}r Mathematik,
%Universit{\"a}t Regensburg,
%93040 Regensburg,
%Germany}
%\email{daniel.schaeppi@ur.de}
\date{}


\title{Monads and their applications -- Sheet 1}




\begin{document}

\SetupExSheets{
 headings=block-subtitle,
}

\pagestyle{empty}
%\maketitle
\section*{Monads and their applications 5}

\begin{question} 
 Let $(S, \sigma)$ be a well-pointed endofunctor of $\ca{D}$. Let $\ca{C}$ be a finitely cocomplete category and let $F \colon \ca{D} \rightarrow \ca{C}$ be left adjoint to $U \colon \ca{C} \rightarrow \ca{D}$. Show that the endofunctor $(S^{\prime},\sigma^{\prime})$ of $\ca{C}$ defined by the pushout diagram
 \[
 \xymatrix{FU \ar[d]_{\varepsilon} \ar[r]^-{F\sigma U} & FSU \ar[d] \\ \id \ar[r]_-{\sigma^{\prime}} & S^{\prime} }
 \]
 in $[\ca{C},\ca{C}]$ is a well-pointed endofunctor and that there is an induced diagram
 \[
\xymatrix{ (S^{\prime},\sigma^{\prime})\mbox{-}\Alg \ar[r]^{\overline{U}} \ar[d] & (S,\sigma)\mbox{-}\Alg \ar[d] \\
\ca{C} \ar[r]_-{U} & \ca{D} }
 \]
 which is a (strict) pullback diagram of categories.
\end{question}

\begin{question}
 Let $\bigl((S_i,\sigma_i)\bigr)_{i \in I}$ be a family of accessible well-pointed  endofunctors of a locally presentable category $\ca{C}$. Let $S$ be the colimit in $[\ca{C},\ca{C}]$ of the ``star-shaped'' diagram $\sigma_i \colon \id_{\ca{C}} \Rightarrow S_i$ (that is, a star with center $\id_{\ca{C}}$). Write $\sigma$ for the induced composite $\id \Rightarrow S_i \Rightarrow S$ (which is by construction independent of $i$). Show that $(S,\sigma)$ is an accessible well-pointed endofunctor and that $(S,\sigma)\mbox{-}\Alg$ is the intersection $\bigcap_{i \in I} (S_i,\sigma_i)\mbox{-} \Alg$.
\end{question}

\begin{question}
 Let $k \colon a \rightarrow b$ be a morphism of a category $\ca{C}$. An object $c \in \ca{C}$ is called \emph{orthogonal} to $k$ if for any $f \colon a \rightarrow c$, there exists a unique dashed arrow making the triangle
 \[
 \xymatrix{ a \ar[r]^-{f} \ar[d]_{k} & c \\ b \ar@{-->}[ru]}
 \]
 commutative. The full subcategory of objects orthogonal to $k$ is denoted by $\{k\}^{\perp} \subseteq \ca{C}$. Similarly, given a set $\Sigma$ of morphisms in $\ca{C}$, we write $\Sigma^{\perp} \subseteq \ca{C}$ for the class of objects which are orthogonal to all the morphisms in $\Sigma$.
 
 \begin{enumerate}
 \item[(a)] Let $\ca{C}$ be locally presentable and let $k \colon a \rightarrow b$ be a morphism in $\ca{C}$. This defines a functor $k \colon [1] \rightarrow \ca{C}$, where $[1]=\{0 \rightarrow 1\}$ denotes the category consisting of a single non-trivial morphism. Show that the right adjoint of the induced adjunction $\Lan_Y k \colon \Set^{[1]^{\op}} \rightleftarrows \ca{C} \colon \widetilde{k}$ is accessible.
 
\item[(b)] Show that there exists an accessible well-pointed endofunctor $(S_k,\sigma_k)$ of $\ca{C}$ such that $(S_k,\sigma_k)\mbox{-}\Alg$ is equal to the full subcategory $\{k\}^{\perp}\subseteq \ca{C}$ of objects orthogonal to $k$. (Hint: apply Exercise~1 to the adjunction of (a).)

\item[(c)] Let $\Sigma$ be a set of morphisms in $\ca{C}$. Show that there exists an accessible well-pointed endofunctor of $\ca{C}$ whose category of algebras is $\Sigma^{\perp} \subseteq \ca{C}$.
 \end{enumerate}
\end{question}


\begin{question}
 Let $\ca{A}$ be a small category and consider a set $\bigl(D^{k} \colon \ca{I}_k \rightarrow \ca{A} \bigr)_{k \in K}$ of diagrams in $\ca{A}$. For each diagram $D^k$, fix a cocone $\kappa_i \colon D^k_i \rightarrow a_k$ in $\ca{A}$. Let $\ca{C} \subseteq [\ca{A}^{\op},\Set]$ be the full subcategory of presheaves $F \colon \ca{A}^{\op} \rightarrow \Set$ with the property that $F\kappa_i \colon Fa_k \rightarrow FD^k_i$ is a limit cone for each $k \in K$. Show that there exists a set $\Sigma$ of morphisms in $[\ca{A}^{\op},\Set]$ such that $\ca{C}=\Sigma^{\perp}$.
\end{question}

\begin{question}[subtitle=(bonus)]
 From the lecture, we know that the category $\ca{C}$ in Exercise~4 is reflective. Show that the composite
 \[
 \xymatrix{ \ca{A} \ar[r]^-{Y} & [\ca{A}^{\op},\Set] \ar[r] & \ca{C}}
 \]
 of the Yoneda embedding and the left adjoint of the inclusion is the universal functor to a cocomplete category which sends the given cones to colimit cones.
\end{question}

\end{document}
