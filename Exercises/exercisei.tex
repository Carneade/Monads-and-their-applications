\documentclass[12pt, a4paper]{article}

\pdfoutput=1

\usepackage{exsheets}

\usepackage{mathrsfs, amsmath, amsthm, amssymb, stmaryrd, enumerate}%
\usepackage{tikz}%
\usepackage[linktocpage]{hyperref}%
%\usepackage{cleveref}
\usepackage[all]{xy}%
\xyoption{2cell}%
\UseAllTwocells%

\newdir{t>}{{\UseTips\dir{>}}}

%\creflabelformat{enumi}{(#2#1#3)}

%\hypersetup{
% pdfauthor={Daniel Schaeppi},
% pdfkeywords={Pushouts} {algebraic stacks} {Quasi-coherent sheaves} {weakly Tannakian categories}
%}


%********************************* MACROS ************************************%

\DeclareMathOperator{\Ob}{Ob}
\DeclareMathOperator{\id}{id}
\DeclareMathOperator{\el}{el}
\DeclareMathOperator{\op}{op}
\DeclareMathOperator{\Lan}{Lan}
\DeclareMathOperator{\Ran}{Ran}
\DeclareMathOperator{\Mod}{\mathbf{Mod}}
\DeclareMathOperator{\Rep}{Rep}
\DeclareMathOperator{\Vect}{\mathbf{Vect}}
\DeclareMathOperator{\Coalg}{\mathbf{Coalg}}
\DeclareMathOperator{\fgp}{fgp}
\DeclareMathOperator{\fp}{fp}
\DeclareMathOperator{\fd}{fd}
\DeclareMathOperator{\Hom}{Hom}
\DeclareMathOperator{\End}{End}
\DeclareMathOperator{\Spec}{Spec}

\DeclareMathOperator{\Tor}{Tor}
\DeclareMathOperator{\reg}{reg}
\DeclareMathOperator{\flt}{flat}


\DeclareMathOperator{\rk}{rk}
\DeclareMathOperator{\Sym}{Sym}
\DeclareMathOperator{\sgn}{sgn}
\DeclareMathOperator{\coev}{coev}
\DeclareMathOperator{\ev}{ev}

\DeclareMathOperator{\fl}{fin. \ell}

\DeclareMathOperator{\fin}{fin}
\DeclareMathOperator{\fg}{fg}
\DeclareMathOperator{\idem}{idem}
\DeclareMathOperator{\Modtwo}{\mathcal{M}}

\DeclareMathOperator{\iso}{iso}

\DeclareMathOperator{\LFdi}{\LF^{\rk d}_{\iso}}

\DeclareMathOperator{\dom}{dom}
\DeclareMathOperator{\Sh}{Sh}
\DeclareMathOperator{\Rex}{\mathbf{Rex}}

\DeclareMathOperator{\SymMonCat}{SymMonCat}


\DeclareMathOperator{\PsCone}{PsCone}


\DeclareMathOperator{\Ind}{Ind}
\DeclareMathOperator{\Span}{\mathbf{Span}}
\DeclareMathOperator{\Cospan}{\mathbf{Cospan}}
\DeclareMathOperator{\EM}{EM}

\DeclareMathOperator{\Aff}{\mathbf{Aff}}
\DeclareMathOperator{\Alg}{\mathbf{Alg}}
\DeclareMathOperator{\Coh}{\mathbf{Coh}}
\DeclareMathOperator{\QCoh}{\mathbf{QCoh}}
\DeclareMathOperator{\VB}{\mathbf{VB}}

\DeclareMathOperator{\BGL}{\mathrm{BGL}}

\DeclareMathOperator{\Cov}{Cov}
\DeclareMathOperator{\fpqc}{\mathit{fpqc}}

\DeclareMathOperator{\Aut}{Aut}
\DeclareMathOperator{\lax}{lax}
\DeclareMathOperator{\Tors}{Tors}


\DeclareMathOperator{\Map}{Map}
\DeclareMathOperator{\target}{target}
\DeclareMathOperator{\cospan}{Cospan}
\DeclareMathOperator{\CommAlg}{\mathbf{CommAlg}}
\DeclareMathOperator{\coop}{coop}
\DeclareMathOperator{\Coact}{Coact}
\DeclareMathOperator{\Gray}{\mathbf{Gray}}
\DeclareMathOperator{\PsMon}{\mathbf{PsMon}}
\DeclareMathOperator{\BrPsMon}{\mathbf{BrPsMon}}
\DeclareMathOperator{\SymPsMon}{\mathbf{SymPsMon}}
\DeclareMathOperator{\MonComon}{\mathbf{MonComon}}
\DeclareMathOperator{\BrMonComon}{\mathbf{BrMonComon}}
\DeclareMathOperator{\SymMonComon}{\mathbf{SymMonComon}}
\DeclareMathOperator{\Psa}{\mathbf{Psa}}

\DeclareMathOperator{\pr}{pr}

\DeclareMathOperator{\len}{\ell}
\DeclareMathOperator{\proj}{proj}
\DeclareMathOperator{\Fil}{Fil}
\DeclareMathOperator{\MF}{MF}
\DeclareMathOperator{\colim}{colim}
\DeclareMathOperator{\Er}{Er}

\DeclareMathOperator{\Ps}{\mathbf{Ps}}

\DeclareMathOperator{\Cocts}{\mathbf{Cocts}}
\DeclareMathOperator{\Lex}{\mathbf{Lex}}
\DeclareMathOperator{\cat}{\mathbf{cat}}
\DeclareMathOperator{\Cat}{\mathbf{Cat}}
\DeclareMathOperator{\Bicat}{\mathbf{Bicat}}
\DeclareMathOperator{\Tricat}{\mathbf{Tricat}}
\DeclareMathOperator{\Gpd}{\mathbf{Gpd}}
\DeclareMathOperator{\Mky}{\mathbf{Mky}}
\DeclareMathOperator{\CAT}{\mathbf{CAT}}
\DeclareMathOperator{\Set}{\mathbf{Set}}
\DeclareMathOperator{\Ab}{\mathbf{Ab}}
\DeclareMathOperator{\CGTop}{\mathbf{CGTop}}
\DeclareMathOperator{\Mon}{\mathbf{Mon}}
\DeclareMathOperator{\Comon}{\mathbf{Comon}}
\DeclareMathOperator{\Comod}{\mathbf{Comod}}

\DeclareMathOperator{\Nat}{\mathrm{Nat}}
\DeclareMathOperator{\Fun}{\mathrm{Fun}}
\DeclareMathOperator{\LF}{\mathrm{LF}}

\DeclareMathOperator{\CAlg}{\mathrm{CAlg}}

\newcommand{\ca}[1]{\mathscr{#1}}
\newcommand{\VNat}{\ca{V}\mbox{-}\Nat}
\newcommand{\Vcat}{\ca{V}\mbox{-}\Cat}
\newcommand{\VCAT}{\ca{V}\mbox{-}\CAT}
\newcommand{\VCATlp}{\ca{V}\mbox{-}\CAT_{\mathrm{lp}}}
\newcommand{\VCATc}{\ca{V}^{\prime}\mbox{-}\Cat_{\mathrm{c}}}

\newcommand{\Catfc}{\Cat_{\mathrm{fc}}}
\newcommand{\VCatfc}{\ca{V}\mbox{-}\Cat_{\mathrm{fc}}}
\newcommand{\kten}{\mathop{\boxtimes_{\mathrm{fc}}}}

\newcommand{\Prs}[1]{\mathcal{P}\ca{#1}}
\newcommand{\Bimod}[1]{{_\ca{#1}}{\mathcal{M}}{_\ca{#1}}}
\newcommand{\modules}[2]{{_\ca{#1}}{\mathcal{M}}{_\ca{#2}}}
\newcommand{\CC}[1]{\mathbf{Comon}\left(\Bimod{#1}\right)}

\DeclareMathOperator{\U}{O}

\newcommand{\ubar}[1]{\underline{#1\mkern-4mu}\mkern4mu }

\newcommand{\dual}[1]{{#1}^{\circ}}
\newcommand{\ldual}[1]{{#1}^{\vee}}


\newcommand{\ten}[1]{\mathop{{\otimes}_{#1}}}
\newcommand{\tenl}[1]{\mathop{{}_{#1}{\otimes}}}
\newcommand{\tenlr}[2]{\mathop{{}_{#1}{\otimes}_{#2}}}

\newcommand{\boxten}[1]{\mathop{{\boxtimes}_{#1}}}

\newcommand{\pb}[1]{\mathop{{\times}_{#1}}}
\newcommand{\po}[1]{\mathop{{+}_{#1}}}


\newcommand{\defl}{\mathrel{\mathop:}=}


% THEOREM ENVIRONMENTS

\theoremstyle{plain}
\newtheorem{thm}{Theorem}[subsection]
\newtheorem*{thm*}{Theorem}
\newtheorem{prop}[thm]{Proposition}
\newtheorem{lemma}[thm]{Lemma}
\newtheorem{cor}[thm]{Corollary}

\theoremstyle{definition}
\newtheorem{example}[thm]{Example}
\newtheorem{rmk}[thm]{Remark}
\newtheorem{dfn}[thm]{Definition}
\newtheorem{notation}[thm]{Notation}

\newtheoremstyle{citing}{}{}{\itshape}{}{\bfseries}{.}{ }{\thmnote{#3}}
\theoremstyle{citing}
\newtheorem{cit}{}

\newtheoremstyle{citingdfn}{}{}{}{}{\bfseries}{.}{ }{\thmnote{#3}}
\theoremstyle{citingdfn}
\newtheorem{citdfn}{}


\numberwithin{equation}{section}

%\keywords{Fiber functors, Tannakian categories}
%\subjclass[2010]{14A20, 18D10}

%\author{Daniel Sch\"appi}
%\thanks{This research was supported by the DFG grant: SFB 1085 ``Higher invariants.''}
%\address{Fakult{\"a}t f{\"u}r Mathematik,
%Universit{\"a}t Regensburg,
%93040 Regensburg,
%Germany}
%\email{daniel.schaeppi@ur.de}
\date{}


\title{Monads and their applications -- Sheet 1}




\begin{document}

\SetupExSheets{
 headings=block-subtitle,
}

\pagestyle{empty}
%\maketitle
\section*{Monads and their applications 1}

\begin{question} 
  Let $(T, \mu, \eta)$ be a monad on $\ca{C}$ and let $(A,\alpha)$ be a $T$-algebra. Show that for any isomorphism $g \colon B \rightarrow A$ there exists a unique $T$-algebra structure $\beta \colon TB \rightarrow B$ on $B$ such that $g$ is a morphism of $T$-algebras $(B,\beta) \rightarrow (A,\alpha)$.
\end{question}

\begin{question}
 A monad $(T,  \mu, \eta)$ on $\ca{C}$ is called \emph{idempotent} if the multiplication $\mu \colon T^2 \Rightarrow T$ is an isomorphism. Show that for any idempotent monad $T \colon \ca{C} \rightarrow \ca{C}$ and any object $A \in \ca{C}$, there exists at most one $T$-algebra structure on $A$. Moreover, show that for any pair of $T$-algebras $(A, \alpha)$ and $(B,\beta)$, every morphism  $f \colon A \rightarrow B$ in $\ca{C}$ is a morphism of $T$-algebras (in other words, the forgetful functor $U^{T} \colon T\mbox{-}\Alg \rightarrow \ca{C}$ is full).
\end{question}

\begin{question}
 Let $R$ be a commutative ring. Show that the functor $\Mod_R \rightarrow \Mod_R$, which sends $M$ to $\oplus_{n \in \mathbb{N}} M^{\otimes n}$ can be endowed with the structure of a monad whose category of algebras is (isomorphic to) the category of $R$-algebras. Find a monad on $\Mod_R$ whose category of algebras is the category of \emph{commutative} $R$-algebras.
\end{question}

\begin{question}[subtitle=(equivalent characterizations of adjunctions)]
 Let $F \colon \ca{C} \rightarrow \ca{D}$ and $U \colon \ca{D} \rightarrow \ca{C}$ be two functors.
 
\begin{enumerate}
\item[(a)] Show that there is a bijection between the set of pairs natural transformations $\eta \colon \id_{\ca{D}} \Rightarrow UF$ and $\varepsilon \colon FU \rightarrow \id_{\ca{C}}$ which satisfy the triangle identites ($\varepsilon F \cdot F\eta=1_F$, $U \varepsilon \cdot \eta U=1_U$) on the one hand and the set of natural isomorphisms
\[
\varphi \colon \ca{D}(F-,-) \Rightarrow \ca{C}(-,U-) \colon \ca{D}^{\op} \times \ca{C} \rightarrow \Set
\]
on the other.
\item[(b)]
 Assume that for each $c \in \ca{C}$, the functor $\ca{C}(c,U-)$ is representable. Suppose that there is an explicit choice of representing object $d_c \in \ca{D}$, that is, a choice of a natural isomorphism $\ca{D}(d_c,-) \cong \ca{C}(c,U-)$. Show that there exists a functor $G \colon \ca{C} \rightarrow \ca{D}$ with $Gc=d_c$ such that $G$ is left adjoint to $U$.

\item[(c)] Show that if both $F$ and $G$ are left adjoint to $U$, then there exists a natural isomorphism $F \cong G$.
\end{enumerate}

\end{question}

%\SetupExSheets{counter-format=se~$\ast$}
\begin{question}[subtitle=(bonus)]
 
\end{question}
 An \emph{ultrafilter} on a set $X$ is a set $\ca{F}$ of subsets of $X$ such that the following axiom holds: for all subsets $A \subseteq X$, $A$ belongs to $F$ if and only if for all $B_1, \ldots, B_n \in \ca{F}$, the intersection $A \cap B_1 \cap \ldots \cap B_n$ is non-empty. The \emph{principal ultrafilter} of $x \in X$ is $\ca{F}_x = \{A \subseteq X \vert x \in A\}$. Given a subset $A \subseteq X$, we write $[A]$ for the set of ultrafilters $\ca{F}$ which contain $A$. We write $UX$ for the set of ultrafilters on $X$. Given a function $f \colon X \rightarrow Y$ and an ultrafilter $\ca{F}$ on $X$, we call
 \[
 f_\ast \ca{F} \defl \{B \subseteq Y \vert f^{-1}(B) \in \ca{F} \}
 \]
  the pushforward of $\ca{F}$.
 
 \begin{enumerate}
 \item[(a)] Show that $UX$ defines an endofunctor of $\Set$ via the pushforward.
 \item[(b)] Show that $\eta_X \colon X \rightarrow UX$, $x \mapsto \ca{F}_x$ and $\mu_X \colon UUX \rightarrow UX$ defined by
 \[
   \mu(\ca{F})=\{A \subseteq X \vert [A] \in \ca{F} \}
 \]
 endow $U$ with the structure of a monad.
 
 \item[(c)]
  Let $\xi \colon UX \rightarrow X$ be an algebra for the ultrafilter monad. We call a subset $U \subseteq X$ \emph{open} if
  \[
  \forall x \in X \; \forall \ca{F} \in UX \colon (x \in U \; \text{and} \; \xi(\ca{F})=x) \Rightarrow  U \in \ca{F}
  \]
  holds. Show that these open sets form a topology on $X$.
\end{enumerate}  
\end{document}
