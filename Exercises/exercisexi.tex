\documentclass[12pt, a4paper]{article}

\pdfoutput=1

\usepackage{exsheets}

\usepackage{mathrsfs, amsmath, amsthm, amssymb, stmaryrd, enumerate}%
\usepackage{tikz}%
\usepackage[linktocpage]{hyperref}%
%\usepackage{cleveref}
\usepackage[all]{xy}%
\xyoption{2cell}%
\UseAllTwocells%

\newdir{t>}{{\UseTips\dir{>}}}

%\creflabelformat{enumi}{(#2#1#3)}

%\hypersetup{
% pdfauthor={Daniel Schaeppi},
% pdfkeywords={Pushouts} {algebraic stacks} {Quasi-coherent sheaves} {weakly Tannakian categories}
%}


%********************************* MACROS ************************************%

\DeclareMathOperator{\Ob}{Ob}
\DeclareMathOperator{\id}{id}
\DeclareMathOperator{\el}{el}
\DeclareMathOperator{\op}{op}
\DeclareMathOperator{\Lan}{Lan}
\DeclareMathOperator{\Ran}{Ran}
\DeclareMathOperator{\Mod}{\mathbf{Mod}}
\DeclareMathOperator{\Rep}{Rep}
\DeclareMathOperator{\Vect}{\mathbf{Vect}}
\DeclareMathOperator{\Coalg}{\mathbf{Coalg}}
\DeclareMathOperator{\fgp}{fgp}
\DeclareMathOperator{\fp}{fp}
\DeclareMathOperator{\fd}{fd}
\DeclareMathOperator{\Hom}{Hom}
\DeclareMathOperator{\End}{End}
\DeclareMathOperator{\Spec}{Spec}

\DeclareMathOperator{\Tor}{Tor}
\DeclareMathOperator{\reg}{reg}
\DeclareMathOperator{\flt}{flat}


\DeclareMathOperator{\rk}{rk}
\DeclareMathOperator{\Sym}{Sym}
\DeclareMathOperator{\sgn}{sgn}
\DeclareMathOperator{\coev}{coev}
\DeclareMathOperator{\ev}{ev}

\DeclareMathOperator{\fl}{fin. \ell}

\DeclareMathOperator{\fin}{fin}
\DeclareMathOperator{\fg}{fg}
\DeclareMathOperator{\idem}{idem}
\DeclareMathOperator{\Modtwo}{\mathcal{M}}

\DeclareMathOperator{\iso}{iso}

\DeclareMathOperator{\LFdi}{\LF^{\rk d}_{\iso}}

\DeclareMathOperator{\dom}{dom}
\DeclareMathOperator{\Sh}{Sh}
\DeclareMathOperator{\Rex}{\mathbf{Rex}}

\DeclareMathOperator{\SymMonCat}{SymMonCat}


\DeclareMathOperator{\PsCone}{PsCone}


\DeclareMathOperator{\Ind}{Ind}
\DeclareMathOperator{\Span}{\mathbf{Span}}
\DeclareMathOperator{\Cospan}{\mathbf{Cospan}}
\DeclareMathOperator{\EM}{EM}

\DeclareMathOperator{\Aff}{\mathbf{Aff}}
\DeclareMathOperator{\Alg}{\mathbf{Alg}}
\DeclareMathOperator{\Coh}{\mathbf{Coh}}
\DeclareMathOperator{\QCoh}{\mathbf{QCoh}}
\DeclareMathOperator{\VB}{\mathbf{VB}}

\DeclareMathOperator{\BGL}{\mathrm{BGL}}

\DeclareMathOperator{\Cov}{Cov}
\DeclareMathOperator{\fpqc}{\mathit{fpqc}}

\DeclareMathOperator{\Aut}{Aut}
\DeclareMathOperator{\lax}{lax}
\DeclareMathOperator{\Tors}{Tors}


\DeclareMathOperator{\Map}{Map}
\DeclareMathOperator{\target}{target}
\DeclareMathOperator{\cospan}{Cospan}
\DeclareMathOperator{\CommAlg}{\mathbf{CommAlg}}
\DeclareMathOperator{\coop}{coop}
\DeclareMathOperator{\Coact}{Coact}
\DeclareMathOperator{\Gray}{\mathbf{Gray}}
\DeclareMathOperator{\PsMon}{\mathbf{PsMon}}
\DeclareMathOperator{\BrPsMon}{\mathbf{BrPsMon}}
\DeclareMathOperator{\SymPsMon}{\mathbf{SymPsMon}}
\DeclareMathOperator{\MonComon}{\mathbf{MonComon}}
\DeclareMathOperator{\BrMonComon}{\mathbf{BrMonComon}}
\DeclareMathOperator{\SymMonComon}{\mathbf{SymMonComon}}
\DeclareMathOperator{\Psa}{\mathbf{Psa}}

\DeclareMathOperator{\pr}{pr}

\DeclareMathOperator{\len}{\ell}
\DeclareMathOperator{\proj}{proj}
\DeclareMathOperator{\Fil}{Fil}
\DeclareMathOperator{\MF}{MF}
\DeclareMathOperator{\colim}{colim}
\DeclareMathOperator{\Er}{Er}

\DeclareMathOperator{\Ps}{\mathbf{Ps}}

\DeclareMathOperator{\Cocts}{\mathbf{Cocts}}
\DeclareMathOperator{\Lex}{\mathbf{Lex}}
\DeclareMathOperator{\cat}{\mathbf{cat}}
\DeclareMathOperator{\Cat}{\mathbf{Cat}}
\DeclareMathOperator{\Bicat}{\mathbf{Bicat}}
\DeclareMathOperator{\Tricat}{\mathbf{Tricat}}
\DeclareMathOperator{\Gpd}{\mathbf{Gpd}}
\DeclareMathOperator{\Mky}{\mathbf{Mky}}
\DeclareMathOperator{\CAT}{\mathbf{CAT}}
\DeclareMathOperator{\Set}{\mathbf{Set}}
\DeclareMathOperator{\Ab}{\mathbf{Ab}}
\DeclareMathOperator{\CGTop}{\mathbf{CGTop}}
\DeclareMathOperator{\Mon}{\mathbf{Mon}}
\DeclareMathOperator{\Comon}{\mathbf{Comon}}
\DeclareMathOperator{\Comod}{\mathbf{Comod}}

\DeclareMathOperator{\Nat}{\mathrm{Nat}}
\DeclareMathOperator{\Fun}{\mathrm{Fun}}
\DeclareMathOperator{\LF}{\mathrm{LF}}

\DeclareMathOperator{\CAlg}{\mathrm{CAlg}}

\newcommand{\ca}[1]{\mathscr{#1}}
\newcommand{\VNat}{\ca{V}\mbox{-}\Nat}
\newcommand{\Vcat}{\ca{V}\mbox{-}\Cat}
\newcommand{\VCAT}{\ca{V}\mbox{-}\CAT}
\newcommand{\VCATlp}{\ca{V}\mbox{-}\CAT_{\mathrm{lp}}}
\newcommand{\VCATc}{\ca{V}^{\prime}\mbox{-}\Cat_{\mathrm{c}}}

\newcommand{\Catfc}{\Cat_{\mathrm{fc}}}
\newcommand{\VCatfc}{\ca{V}\mbox{-}\Cat_{\mathrm{fc}}}
\newcommand{\kten}{\mathop{\boxtimes_{\mathrm{fc}}}}

\newcommand{\Prs}[1]{\mathcal{P}\ca{#1}}
\newcommand{\Bimod}[1]{{_\ca{#1}}{\mathcal{M}}{_\ca{#1}}}
\newcommand{\modules}[2]{{_\ca{#1}}{\mathcal{M}}{_\ca{#2}}}
\newcommand{\CC}[1]{\mathbf{Comon}\left(\Bimod{#1}\right)}

\DeclareMathOperator{\U}{O}

\newcommand{\ubar}[1]{\underline{#1\mkern-4mu}\mkern4mu }

\newcommand{\dual}[1]{{#1}^{\circ}}
\newcommand{\ldual}[1]{{#1}^{\vee}}


\newcommand{\ten}[1]{\mathop{{\otimes}_{#1}}}
\newcommand{\tenl}[1]{\mathop{{}_{#1}{\otimes}}}
\newcommand{\tenlr}[2]{\mathop{{}_{#1}{\otimes}_{#2}}}

\newcommand{\boxten}[1]{\mathop{{\boxtimes}_{#1}}}

\newcommand{\pb}[1]{\mathop{{\times}_{#1}}}
\newcommand{\po}[1]{\mathop{{+}_{#1}}}


\newcommand{\defl}{\mathrel{\mathop:}=}


% THEOREM ENVIRONMENTS

\theoremstyle{plain}
\newtheorem{thm}{Theorem}[subsection]
\newtheorem*{thm*}{Theorem}
\newtheorem{prop}[thm]{Proposition}
\newtheorem{lemma}[thm]{Lemma}
\newtheorem{cor}[thm]{Corollary}

\theoremstyle{definition}
\newtheorem{example}[thm]{Example}
\newtheorem{rmk}[thm]{Remark}
\newtheorem{dfn}[thm]{Definition}
\newtheorem{notation}[thm]{Notation}

\newtheoremstyle{citing}{}{}{\itshape}{}{\bfseries}{.}{ }{\thmnote{#3}}
\theoremstyle{citing}
\newtheorem{cit}{}

\newtheoremstyle{citingdfn}{}{}{}{}{\bfseries}{.}{ }{\thmnote{#3}}
\theoremstyle{citingdfn}
\newtheorem{citdfn}{}


\numberwithin{equation}{section}

%\keywords{Fiber functors, Tannakian categories}
%\subjclass[2010]{14A20, 18D10}

%\author{Daniel Sch\"appi}
%\thanks{This research was supported by the DFG grant: SFB 1085 ``Higher invariants.''}
%\address{Fakult{\"a}t f{\"u}r Mathematik,
%Universit{\"a}t Regensburg,
%93040 Regensburg,
%Germany}
%\email{daniel.schaeppi@ur.de}
\date{}


\title{Monads and their applications -- Sheet 1}




\begin{document}

\SetupExSheets{
 headings=block-subtitle,
}

\pagestyle{empty}
%\maketitle
\section*{Monads and their applications 11}

\begin{question} 
 Let $\ca{V}$ be the category $\mathrm{Ch}(\Ab)$ of chain complexes over abelian groups (``differential graded $\mathbb{Z}$-modules''). The tensor product of $X_{\bullet}$ and $Y_{\bullet}$ is given in degree $n$ by $\oplus_{i+j=n} X_i \otimes Y_j$, with differential $d(x\otimes y)=dx \otimes y +(-1)^{i} x\otimes dy$ for $x\otimes y \in X_i \otimes Y_j$. Show that this is a symmetric monoidal closed category and explicitly describe the self-enrichment of $\ca{V}$ (specifically,  the composition morphism).
 \end{question}

\begin{question}
 Let $\ca{C}$ be a $\ca{V}$-category with powers, $T \colon \ca{C} \rightarrow \ca{C}$ a $\ca{V}$-monad. Show that the forgetful $\ca{V}$-functor $T\mbox{-}\Alg \rightarrow \ca{C}$ creates powers (that is, for every $T$-algebra $(A,a)$ and $V \in \ca{V}$ one can put a unique $T$-algebra structure on $A^{V}$ making it a power of $(A,a)$ in $T\mbox{-}\Alg$).
\end{question}

\begin{question}
 Let $t \colon C \rightarrow C$ be a monad in the 2-category $\ca{K}$. A $t$-action on a 1-cell $g \colon A \rightarrow C$ is a 2-cell $\alpha \colon tg \Rightarrow g$ satisfiying the laws for a $t$-algebra. This defines a 2-functor
 \[
 t\mbox{-}\mathrm{act}(-) \colon \ca{K}^{\op} \rightarrow \Cat
 \]
 and we say that the \emph{Eilenberg--Moore} object of $t$ exists if this 2-functor is representable, that is, there exists a \emph{universal} $t$-action $(u,\alpha) \colon C_t \rightarrow C$.
 
 \begin{enumerate}
 \item[(a)] Show that $U \colon T\mbox{-}\Alg \rightarrow \ca{C}$ is an Eilenberg--Moore object in $\ca{V}\mbox{-}\Cat$.
 
 \item[(b)] Given any Eilenberg--More object $u \colon C_t \rightarrow C$, show that there exists a left adjoint $f$ of $u$ such that the monad $uf$ is $t$. (Hint: define a suitable $t$-action on $t$ itself).
 \end{enumerate}
\end{question}


\begin{question}
 \begin{enumerate}
  \item[(a)] Let $\ca{V}=\Set$ and consider the discrete category $2=\{0,1\}$ on two objects. A weight $W$ on this category amounts to a choice of two sets. What is the $W$-weighted colimit on a diagram $2 \rightarrow \ca{C}$?
  
  \item[(b)] Let $\ca{V}=\Ab$ and let $R$ be a ring, considered as a 1-object $\ca{V}$-category. Let $\ca{C}=\Ab$ as well. Show that weighted colimits over $R$ correspond to the usual tensor product of a right and a left $R$-module.
  
  \item[(c)] Let $\ca{V}=\Cat$ and consider the category $\ca{I}=\xymatrix{0 \ar[r] & 2 &  1 \ar[l]}$. Let $W$ be the weight with $W(0)=\ast$, $W(1)=\ast$, and $W(2)=[1]$, with the morphisms $W(i) \rightarrow [1]$ picking out $i \in [1]=\{0 \rightarrow 1\}$. Show that $W$-weighted limits in a 2-category $\ca{K}$ are precisely comma-objects.
 \end{enumerate}
 \end{question}

\begin{question}[subtitle=(bonus)]
 Show that $\prod_{j \in J} \ca{V}$ is the free cocomplete $\ca{V}$-cateogry on the discrete category $j$. More precisely, given a cocomplete $\ca{V}$-categories $\ca{C}$, $\ca{D}$, write $\Cocts_0[\ca{C},\ca{D}]$ for the category of cocontinuous $\ca{V}$-functors and $\ca{V}$-natural transformations. Show that the functor $ \Cocts_0[\prod\nolimits_{j \in J} \ca{V},\ca{C}] \rightarrow \prod_{j \in J}\ca{C}_0$ given by $F \mapsto (FI_j)_{j \in J}$ is an equivalence of categories. Here $I_j$ stands for the object which is given by $I$ in degree $j$ and by the initial object everywhere else. You need to be careful when showing that the functor is full! 
 
 If one checks all the requirements by only referring to copowers, $\ca{V}$-coproducts, and $\ca{V}$-coequalizers, (the latter is not even necessary), then one can use this to show that cocontinuous endofunctors of the product are equivalent, as a monoidal category, to $\ca{V}$-matrices on $J$. This can then be used to give a rigorous proof that any $\ca{V}$-category with object set $J$ gives rise to the cocontinous $\ca{V}$-monad $T$ defined in the lecture.
\end{question}



\end{document}
