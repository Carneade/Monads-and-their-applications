\documentclass[12pt, a4paper]{article}

\pdfoutput=1

\usepackage{exsheets}

\usepackage{mathrsfs, amsmath, amsthm, amssymb, stmaryrd, enumerate}%
\usepackage{tikz}%
\usepackage[linktocpage]{hyperref}%
%\usepackage{cleveref}
\usepackage[all]{xy}%
\xyoption{2cell}%
\UseAllTwocells%

\newdir{t>}{{\UseTips\dir{>}}}

%\creflabelformat{enumi}{(#2#1#3)}

%\hypersetup{
% pdfauthor={Daniel Schaeppi},
% pdfkeywords={Pushouts} {algebraic stacks} {Quasi-coherent sheaves} {weakly Tannakian categories}
%}


%********************************* MACROS ************************************%

\DeclareMathOperator{\Ob}{Ob}
\DeclareMathOperator{\id}{id}
\DeclareMathOperator{\el}{el}
\DeclareMathOperator{\op}{op}
\DeclareMathOperator{\Lan}{Lan}
\DeclareMathOperator{\Ran}{Ran}
\DeclareMathOperator{\Mod}{\mathbf{Mod}}
\DeclareMathOperator{\Rep}{Rep}
\DeclareMathOperator{\Vect}{\mathbf{Vect}}
\DeclareMathOperator{\Coalg}{\mathbf{Coalg}}
\DeclareMathOperator{\fgp}{fgp}
\DeclareMathOperator{\fp}{fp}
\DeclareMathOperator{\fd}{fd}
\DeclareMathOperator{\Hom}{Hom}
\DeclareMathOperator{\End}{End}
\DeclareMathOperator{\Spec}{Spec}

\DeclareMathOperator{\Tor}{Tor}
\DeclareMathOperator{\reg}{reg}
\DeclareMathOperator{\flt}{flat}


\DeclareMathOperator{\rk}{rk}
\DeclareMathOperator{\Sym}{Sym}
\DeclareMathOperator{\sgn}{sgn}
\DeclareMathOperator{\coev}{coev}
\DeclareMathOperator{\ev}{ev}

\DeclareMathOperator{\fl}{fin. \ell}

\DeclareMathOperator{\fin}{fin}
\DeclareMathOperator{\fg}{fg}
\DeclareMathOperator{\idem}{idem}
\DeclareMathOperator{\Modtwo}{\mathcal{M}}

\DeclareMathOperator{\iso}{iso}

\DeclareMathOperator{\LFdi}{\LF^{\rk d}_{\iso}}

\DeclareMathOperator{\dom}{dom}
\DeclareMathOperator{\Sh}{Sh}
\DeclareMathOperator{\Rex}{\mathbf{Rex}}

\DeclareMathOperator{\SymMonCat}{SymMonCat}


\DeclareMathOperator{\PsCone}{PsCone}


\DeclareMathOperator{\Ind}{Ind}
\DeclareMathOperator{\Span}{\mathbf{Span}}
\DeclareMathOperator{\Cospan}{\mathbf{Cospan}}
\DeclareMathOperator{\EM}{EM}

\DeclareMathOperator{\Aff}{\mathbf{Aff}}
\DeclareMathOperator{\Alg}{\mathbf{Alg}}
\DeclareMathOperator{\Coh}{\mathbf{Coh}}
\DeclareMathOperator{\QCoh}{\mathbf{QCoh}}
\DeclareMathOperator{\VB}{\mathbf{VB}}

\DeclareMathOperator{\BGL}{\mathrm{BGL}}

\DeclareMathOperator{\Cov}{Cov}
\DeclareMathOperator{\fpqc}{\mathit{fpqc}}

\DeclareMathOperator{\Aut}{Aut}
\DeclareMathOperator{\lax}{lax}
\DeclareMathOperator{\Tors}{Tors}


\DeclareMathOperator{\Map}{Map}
\DeclareMathOperator{\target}{target}
\DeclareMathOperator{\cospan}{Cospan}
\DeclareMathOperator{\CommAlg}{\mathbf{CommAlg}}
\DeclareMathOperator{\coop}{coop}
\DeclareMathOperator{\Coact}{Coact}
\DeclareMathOperator{\Gray}{\mathbf{Gray}}
\DeclareMathOperator{\PsMon}{\mathbf{PsMon}}
\DeclareMathOperator{\BrPsMon}{\mathbf{BrPsMon}}
\DeclareMathOperator{\SymPsMon}{\mathbf{SymPsMon}}
\DeclareMathOperator{\MonComon}{\mathbf{MonComon}}
\DeclareMathOperator{\BrMonComon}{\mathbf{BrMonComon}}
\DeclareMathOperator{\SymMonComon}{\mathbf{SymMonComon}}
\DeclareMathOperator{\Psa}{\mathbf{Psa}}

\DeclareMathOperator{\pr}{pr}

\DeclareMathOperator{\len}{\ell}
\DeclareMathOperator{\proj}{proj}
\DeclareMathOperator{\Fil}{Fil}
\DeclareMathOperator{\MF}{MF}
\DeclareMathOperator{\colim}{colim}
\DeclareMathOperator{\Er}{Er}

\DeclareMathOperator{\Ps}{\mathbf{Ps}}

\DeclareMathOperator{\Cocts}{\mathbf{Cocts}}
\DeclareMathOperator{\Lex}{\mathbf{Lex}}
\DeclareMathOperator{\cat}{\mathbf{cat}}
\DeclareMathOperator{\Cat}{\mathbf{Cat}}
\DeclareMathOperator{\Bicat}{\mathbf{Bicat}}
\DeclareMathOperator{\Tricat}{\mathbf{Tricat}}
\DeclareMathOperator{\Gpd}{\mathbf{Gpd}}
\DeclareMathOperator{\Mky}{\mathbf{Mky}}
\DeclareMathOperator{\CAT}{\mathbf{CAT}}
\DeclareMathOperator{\Set}{\mathbf{Set}}
\DeclareMathOperator{\Ab}{\mathbf{Ab}}
\DeclareMathOperator{\CGTop}{\mathbf{CGTop}}
\DeclareMathOperator{\Mon}{\mathbf{Mon}}
\DeclareMathOperator{\Comon}{\mathbf{Comon}}
\DeclareMathOperator{\Comod}{\mathbf{Comod}}

\DeclareMathOperator{\Nat}{\mathrm{Nat}}
\DeclareMathOperator{\Fun}{\mathrm{Fun}}
\DeclareMathOperator{\LF}{\mathrm{LF}}

\DeclareMathOperator{\CAlg}{\mathrm{CAlg}}

\newcommand{\ca}[1]{\mathscr{#1}}
\newcommand{\VNat}{\ca{V}\mbox{-}\Nat}
\newcommand{\Vcat}{\ca{V}\mbox{-}\Cat}
\newcommand{\VCAT}{\ca{V}\mbox{-}\CAT}
\newcommand{\VCATlp}{\ca{V}\mbox{-}\CAT_{\mathrm{lp}}}
\newcommand{\VCATc}{\ca{V}^{\prime}\mbox{-}\Cat_{\mathrm{c}}}

\newcommand{\Catfc}{\Cat_{\mathrm{fc}}}
\newcommand{\VCatfc}{\ca{V}\mbox{-}\Cat_{\mathrm{fc}}}
\newcommand{\kten}{\mathop{\boxtimes_{\mathrm{fc}}}}

\newcommand{\Prs}[1]{\mathcal{P}\ca{#1}}
\newcommand{\Bimod}[1]{{_\ca{#1}}{\mathcal{M}}{_\ca{#1}}}
\newcommand{\modules}[2]{{_\ca{#1}}{\mathcal{M}}{_\ca{#2}}}
\newcommand{\CC}[1]{\mathbf{Comon}\left(\Bimod{#1}\right)}

\DeclareMathOperator{\U}{O}

\newcommand{\ubar}[1]{\underline{#1\mkern-4mu}\mkern4mu }

\newcommand{\dual}[1]{{#1}^{\circ}}
\newcommand{\ldual}[1]{{#1}^{\vee}}


\newcommand{\ten}[1]{\mathop{{\otimes}_{#1}}}
\newcommand{\tenl}[1]{\mathop{{}_{#1}{\otimes}}}
\newcommand{\tenlr}[2]{\mathop{{}_{#1}{\otimes}_{#2}}}

\newcommand{\boxten}[1]{\mathop{{\boxtimes}_{#1}}}

\newcommand{\pb}[1]{\mathop{{\times}_{#1}}}
\newcommand{\po}[1]{\mathop{{+}_{#1}}}


\newcommand{\defl}{\mathrel{\mathop:}=}


% THEOREM ENVIRONMENTS

\theoremstyle{plain}
\newtheorem{thm}{Theorem}[subsection]
\newtheorem*{thm*}{Theorem}
\newtheorem{prop}[thm]{Proposition}
\newtheorem{lemma}[thm]{Lemma}
\newtheorem{cor}[thm]{Corollary}

\theoremstyle{definition}
\newtheorem{example}[thm]{Example}
\newtheorem{rmk}[thm]{Remark}
\newtheorem{dfn}[thm]{Definition}
\newtheorem{notation}[thm]{Notation}

\newtheoremstyle{citing}{}{}{\itshape}{}{\bfseries}{.}{ }{\thmnote{#3}}
\theoremstyle{citing}
\newtheorem{cit}{}

\newtheoremstyle{citingdfn}{}{}{}{}{\bfseries}{.}{ }{\thmnote{#3}}
\theoremstyle{citingdfn}
\newtheorem{citdfn}{}


\numberwithin{equation}{section}

%\keywords{Fiber functors, Tannakian categories}
%\subjclass[2010]{14A20, 18D10}

%\author{Daniel Sch\"appi}
%\thanks{This research was supported by the DFG grant: SFB 1085 ``Higher invariants.''}
%\address{Fakult{\"a}t f{\"u}r Mathematik,
%Universit{\"a}t Regensburg,
%93040 Regensburg,
%Germany}
%\email{daniel.schaeppi@ur.de}
\date{}


\title{Monads and their applications -- Sheet 1}




\begin{document}

\SetupExSheets{
 headings=block-subtitle,
}

\pagestyle{empty}
%\maketitle
\section*{Monads and their applications 3}

\begin{question} 
 Let $F \colon \ca{A} \rightarrow \ca{C}$, $K \colon \ca{A} \rightarrow \ca{B}$ and $L \colon \ca{B} \rightarrow \ca{C}$ be functors. A natural transformation $\eta \colon F \Rightarrow L \circ K$ is said to exhibit $L$ as left Kan extension of $F$ along $K$ if the composite
 \[
 \xymatrix{ [\ca{B}, \ca{C}](L,G) \ar[r]^-{- \circ K} & [\ca{A},\ca{C}](LK,G) \ar[r]^{\eta^{\ast}} & [\ca{A},\ca{C}](F,G)}
 \]
 is a bijection for all functors $G \colon \ca{B} \rightarrow \ca{C}$. If a left Kan extension of $F$ along $K$ exists, then it is unique up to unique natural isomorphism and it is denoted by $\Lan_K F$.
  \begin{enumerate}
   \item[(a)] Show that left adjoints preserve left Kan extensions in the following sense: if $\eta \colon F \Rightarrow LK$ exhibits $L$  as left Kan extension of $F$ along $K$ and $H \colon \ca{C} \rightarrow \ca{D}$ is a left adjoint, then $H\eta$ exhibits $HL$ as left Kan extension of $HF$ along $K$.
   
   \item[(b)] Show that left Kan extensions compose: if $\Lan_K F$ exists and 
   \[
   K^{\prime} \colon \ca{B} \rightarrow \ca{B}^{\prime}
   \]
   is any functor, then $\Lan_{K^{\prime}} \Lan_K F$ exists if and only if $\Lan_{K^{\prime} K} F$ exists. Moreover, show that in this case there is a natural isomorphism $\Lan_{K^{\prime}} \Lan_K F \cong \Lan_{K^{\prime} K} F$.
  \end{enumerate}
\end{question}

\begin{question}
 The notion of \emph{right} Kan extension is dual to left Kan extension: it is given by a universal natural transformation $\gamma \colon RK \Rightarrow F$ and denoted by $\Ran_K F$.
 \begin{enumerate}
 \item[(a)] Let $F \colon \ca{A} \rightarrow \ca{C}$ be a functor such that the right Kan extension 
 \[
 \Ran_F F \colon \ca{C} \rightarrow \ca{C}
 \] 
 of $F$ along itself exists. Show that $\Ran_F F$ has the structure of a monad in a natural way. This monad is called the \emph{codensity monad} of $F$.
 \item[(b)] If $\ca{A}=\ast$ is the terminal category, then giving a functor $F \colon \ca{A} \rightarrow \ca{C}$ amounts to picking an object $c \in \ca{C}$, $c=F(\ast)$. Show that, in this case, $\Ran_F F$ exists if $\ca{C}$ has products. The resulting codensity monad is called the \emph{endomorphism monad} of $c$ and denoted by $\langle c,c\rangle$. 
\end{enumerate}  
\end{question}

\begin{question}
 Let $k$ be a field and $\Vect_{k}$ the category of $k$-vector spaces. Let $\ca{A}=\{k\}$ be the full subcategory on the one-dimensional vector space $k$. Note that every object of $\Vect_k$ is a colimit of some diagram that factors through $\ca{A}$ (since all vector spaces are free).
 
 \begin{enumerate}
 \item[(a)] Show that, nevertheless, the inclusion $\ca{A} \rightarrow \Vect_k$ is \emph{not} dense.
 
 \item[(b)] Let $\ca{B} =\{ k \oplus k\}$ be the full subcategory one the two-dimensional vector space. Show that the inclusion $\ca{B} \rightarrow \Vect_k$ is dense.
 \end{enumerate}
\end{question}


\begin{question}
  Let $\ca{A}$ be a small category and let $Y \colon \ca{A} \rightarrow [\ca{A}^{\op},\Set]$ be the Yoneda embedding. 
 \begin{enumerate} 
 \item[(a)] Use the Yoneda lemma to show that the canonical cocone on $Y \slash F$ exhibits $F$ as colimit of the domain functor $\mathrm{dom} \colon Y \slash F \rightarrow [\ca{A}^{\op},\Set]$, $(\varphi \colon \ca{A}(-,a) \Rightarrow F) \mapsto \ca{A}(-,a)$.
 
\item[(b)] The category $\el(F)$ of elements of $F$ has objects the pairs $(a,x)$ where $a \in \ca{A}$ and $x \in Fa$ and morphisms $(a,x) \rightarrow (b,y)$ the morphisms $f \colon a \rightarrow b$ in $\ca{A}$ which satisfy $Ff(y)=x$. Show that there is an isomorphism $Y \slash F \cong \el(F)^{\op}$.
 \end{enumerate}
\end{question}

\begin{question}[subtitle=(bonus)]
 An object $c \in \ca{C}$ is called \emph{strongly finitely presentable} if the representable functor $\ca{C}(c,-) \colon \ca{C} \rightarrow \Set$ preserves sifted colimits. A cocomplete category $\ca{C}$ is called \emph{locally strongly finitely presentable} if there exists a small dense subcategory $\ca{A}$ of $\ca{C}$ which consists of strongly finitely presentable objects.
 
 \begin{enumerate}
 \item[(a)] Show that finite coproducts of strongly finitely presentable objects are strongly finitely presentable.
 
 \item[(b)] Let $U \colon \ca{D} \rightarrow \ca{C}$ have a left adjoint $F \colon \ca{C} \rightarrow \ca{D}$. Show the following claim: if $U$ preserves sifted colimits, then $F$ preserves strongly finitely presentable objects.
 
 \item[(c)] Let $\ca{C}$ be a strongly finitely presentable category and let $T \colon \ca{C} \rightarrow \ca{C}$ be a monad which commutes with sifted colimits. Show that $T\mbox{-}\Alg$ is locally strongly finitely presentable. (Hint: let $\ca{A}$ be a dense subcategory of $\ca{C}$ consisting of strongly finitely presentable objects. Show that the objects $(Ta,\mu_a)$ form a dense subcategory of $T\mbox{-}\Alg$).
 \end{enumerate}
\end{question}

\end{document}
