\documentclass[12pt, a4paper]{article}

\pdfoutput=1

\usepackage{exsheets}

\usepackage{mathrsfs, amsmath, amsthm, amssymb, stmaryrd, enumerate}%
\usepackage{tikz}%
\usepackage[linktocpage]{hyperref}%
%\usepackage{cleveref}
\usepackage[all]{xy}%
\xyoption{2cell}%
\UseAllTwocells%

\newdir{t>}{{\UseTips\dir{>}}}

%\creflabelformat{enumi}{(#2#1#3)}

%\hypersetup{
% pdfauthor={Daniel Schaeppi},
% pdfkeywords={Pushouts} {algebraic stacks} {Quasi-coherent sheaves} {weakly Tannakian categories}
%}


%********************************* MACROS ************************************%

\DeclareMathOperator{\Ob}{Ob}
\DeclareMathOperator{\id}{id}
\DeclareMathOperator{\el}{el}
\DeclareMathOperator{\op}{op}
\DeclareMathOperator{\Lan}{Lan}
\DeclareMathOperator{\Ran}{Ran}
\DeclareMathOperator{\Mod}{\mathbf{Mod}}
\DeclareMathOperator{\Rep}{Rep}
\DeclareMathOperator{\Vect}{\mathbf{Vect}}
\DeclareMathOperator{\Coalg}{\mathbf{Coalg}}
\DeclareMathOperator{\fgp}{fgp}
\DeclareMathOperator{\fp}{fp}
\DeclareMathOperator{\fd}{fd}
\DeclareMathOperator{\Hom}{Hom}
\DeclareMathOperator{\End}{End}
\DeclareMathOperator{\Spec}{Spec}

\DeclareMathOperator{\Tor}{Tor}
\DeclareMathOperator{\reg}{reg}
\DeclareMathOperator{\flt}{flat}


\DeclareMathOperator{\rk}{rk}
\DeclareMathOperator{\Sym}{Sym}
\DeclareMathOperator{\sgn}{sgn}
\DeclareMathOperator{\coev}{coev}
\DeclareMathOperator{\ev}{ev}

\DeclareMathOperator{\fl}{fin. \ell}

\DeclareMathOperator{\fin}{fin}
\DeclareMathOperator{\fg}{fg}
\DeclareMathOperator{\idem}{idem}
\DeclareMathOperator{\Modtwo}{\mathcal{M}}

\DeclareMathOperator{\iso}{iso}

\DeclareMathOperator{\LFdi}{\LF^{\rk d}_{\iso}}

\DeclareMathOperator{\dom}{dom}
\DeclareMathOperator{\Sh}{Sh}
\DeclareMathOperator{\Rex}{\mathbf{Rex}}

\DeclareMathOperator{\SymMonCat}{SymMonCat}


\DeclareMathOperator{\PsCone}{PsCone}


\DeclareMathOperator{\Ind}{Ind}
\DeclareMathOperator{\Span}{\mathbf{Span}}
\DeclareMathOperator{\Cospan}{\mathbf{Cospan}}
\DeclareMathOperator{\EM}{EM}

\DeclareMathOperator{\Aff}{\mathbf{Aff}}
\DeclareMathOperator{\Alg}{\mathbf{Alg}}
\DeclareMathOperator{\Coh}{\mathbf{Coh}}
\DeclareMathOperator{\QCoh}{\mathbf{QCoh}}
\DeclareMathOperator{\VB}{\mathbf{VB}}

\DeclareMathOperator{\BGL}{\mathrm{BGL}}

\DeclareMathOperator{\Cov}{Cov}
\DeclareMathOperator{\fpqc}{\mathit{fpqc}}

\DeclareMathOperator{\Aut}{Aut}
\DeclareMathOperator{\lax}{lax}
\DeclareMathOperator{\Tors}{Tors}


\DeclareMathOperator{\Map}{Map}
\DeclareMathOperator{\target}{target}
\DeclareMathOperator{\cospan}{Cospan}
\DeclareMathOperator{\CommAlg}{\mathbf{CommAlg}}
\DeclareMathOperator{\coop}{coop}
\DeclareMathOperator{\Coact}{Coact}
\DeclareMathOperator{\Gray}{\mathbf{Gray}}
\DeclareMathOperator{\PsMon}{\mathbf{PsMon}}
\DeclareMathOperator{\BrPsMon}{\mathbf{BrPsMon}}
\DeclareMathOperator{\SymPsMon}{\mathbf{SymPsMon}}
\DeclareMathOperator{\MonComon}{\mathbf{MonComon}}
\DeclareMathOperator{\BrMonComon}{\mathbf{BrMonComon}}
\DeclareMathOperator{\SymMonComon}{\mathbf{SymMonComon}}
\DeclareMathOperator{\Psa}{\mathbf{Psa}}

\DeclareMathOperator{\pr}{pr}

\DeclareMathOperator{\len}{\ell}
\DeclareMathOperator{\proj}{proj}
\DeclareMathOperator{\Fil}{Fil}
\DeclareMathOperator{\MF}{MF}
\DeclareMathOperator{\colim}{colim}
\DeclareMathOperator{\Er}{Er}

\DeclareMathOperator{\Ps}{\mathbf{Ps}}

\DeclareMathOperator{\Cocts}{\mathbf{Cocts}}
\DeclareMathOperator{\Lex}{\mathbf{Lex}}
\DeclareMathOperator{\cat}{\mathbf{cat}}
\DeclareMathOperator{\Cat}{\mathbf{Cat}}
\DeclareMathOperator{\Bicat}{\mathbf{Bicat}}
\DeclareMathOperator{\Tricat}{\mathbf{Tricat}}
\DeclareMathOperator{\Gpd}{\mathbf{Gpd}}
\DeclareMathOperator{\Mky}{\mathbf{Mky}}
\DeclareMathOperator{\CAT}{\mathbf{CAT}}
\DeclareMathOperator{\Set}{\mathbf{Set}}
\DeclareMathOperator{\Ab}{\mathbf{Ab}}
\DeclareMathOperator{\CGTop}{\mathbf{CGTop}}
\DeclareMathOperator{\Mon}{\mathbf{Mon}}
\DeclareMathOperator{\Comon}{\mathbf{Comon}}
\DeclareMathOperator{\Comod}{\mathbf{Comod}}

\DeclareMathOperator{\Nat}{\mathrm{Nat}}
\DeclareMathOperator{\Fun}{\mathrm{Fun}}
\DeclareMathOperator{\LF}{\mathrm{LF}}

\DeclareMathOperator{\CAlg}{\mathrm{CAlg}}

\newcommand{\ca}[1]{\mathscr{#1}}
\newcommand{\VNat}{\ca{V}\mbox{-}\Nat}
\newcommand{\Vcat}{\ca{V}\mbox{-}\Cat}
\newcommand{\VCAT}{\ca{V}\mbox{-}\CAT}
\newcommand{\VCATlp}{\ca{V}\mbox{-}\CAT_{\mathrm{lp}}}
\newcommand{\VCATc}{\ca{V}^{\prime}\mbox{-}\Cat_{\mathrm{c}}}

\newcommand{\Catfc}{\Cat_{\mathrm{fc}}}
\newcommand{\VCatfc}{\ca{V}\mbox{-}\Cat_{\mathrm{fc}}}
\newcommand{\kten}{\mathop{\boxtimes_{\mathrm{fc}}}}

\newcommand{\Prs}[1]{\mathcal{P}\ca{#1}}
\newcommand{\Bimod}[1]{{_\ca{#1}}{\mathcal{M}}{_\ca{#1}}}
\newcommand{\modules}[2]{{_\ca{#1}}{\mathcal{M}}{_\ca{#2}}}
\newcommand{\CC}[1]{\mathbf{Comon}\left(\Bimod{#1}\right)}

\DeclareMathOperator{\U}{O}

\newcommand{\ubar}[1]{\underline{#1\mkern-4mu}\mkern4mu }

\newcommand{\dual}[1]{{#1}^{\circ}}
\newcommand{\ldual}[1]{{#1}^{\vee}}


\newcommand{\ten}[1]{\mathop{{\otimes}_{#1}}}
\newcommand{\tenl}[1]{\mathop{{}_{#1}{\otimes}}}
\newcommand{\tenlr}[2]{\mathop{{}_{#1}{\otimes}_{#2}}}

\newcommand{\boxten}[1]{\mathop{{\boxtimes}_{#1}}}

\newcommand{\pb}[1]{\mathop{{\times}_{#1}}}
\newcommand{\po}[1]{\mathop{{+}_{#1}}}


\newcommand{\defl}{\mathrel{\mathop:}=}


% THEOREM ENVIRONMENTS

\theoremstyle{plain}
\newtheorem{thm}{Theorem}[subsection]
\newtheorem*{thm*}{Theorem}
\newtheorem{prop}[thm]{Proposition}
\newtheorem{lemma}[thm]{Lemma}
\newtheorem{cor}[thm]{Corollary}

\theoremstyle{definition}
\newtheorem{example}[thm]{Example}
\newtheorem{rmk}[thm]{Remark}
\newtheorem{dfn}[thm]{Definition}
\newtheorem{notation}[thm]{Notation}

\newtheoremstyle{citing}{}{}{\itshape}{}{\bfseries}{.}{ }{\thmnote{#3}}
\theoremstyle{citing}
\newtheorem{cit}{}

\newtheoremstyle{citingdfn}{}{}{}{}{\bfseries}{.}{ }{\thmnote{#3}}
\theoremstyle{citingdfn}
\newtheorem{citdfn}{}


\numberwithin{equation}{section}

%\keywords{Fiber functors, Tannakian categories}
%\subjclass[2010]{14A20, 18D10}

%\author{Daniel Sch\"appi}
%\thanks{This research was supported by the DFG grant: SFB 1085 ``Higher invariants.''}
%\address{Fakult{\"a}t f{\"u}r Mathematik,
%Universit{\"a}t Regensburg,
%93040 Regensburg,
%Germany}
%\email{daniel.schaeppi@ur.de}
\date{}


\title{Monads and their applications -- Sheet 1}




\begin{document}

\SetupExSheets{
 headings=block-subtitle,
}

\pagestyle{empty}
%\maketitle
\section*{Monads and their applications 12}

\begin{question} 
 Let $\ca{V}$ be a cosmos, $A$ and $B$ monoids in $\ca{V}$. The category $\ca{V}_A$ of left $A$-modules is precisely the functor category $[A,\ca{V}]$. Show that the category of left $\ca{V}$-adjoints $\ca{V}_A \rightarrow \ca{V}_B$ is equivalent to the category of $A$-$B$-bimodules (Hint: use the theory of free cocompletions). Apply this to the case of equivalences to show that the module categories are equivalent as $\ca{V}$-categories if and only if there exists an $A$-$B$ bimodule $M$ and a $B$-$A$-bimodule $N$ with $N\ten{B}M \cong A$ and $M \ten{A} N \cong B$ (this is called \emph{Morita equivalence} of monoids).
 \end{question}

\begin{question}
 Recall that an Eilenberg--Moore object (EM-object for short) represents actions of a monoad in a 2-category $\ca{K}$. A \emph{Kleisli-object} is the dual notion, that is, an EM-object in $\ca{K}^{\op}$: given a monad $t \colon C \rightarrow C$, the Kleisli-object is the universal morphism $f \colon C \rightarrow C^t$ with an action $ft \Rightarrow f$.
 
 Show that $\CAT$ has Kleisli-objects, described as follows: for a monad $T \colon \ca{C} \rightarrow \ca{C}$, the category $\ca{C}^{T}$ has the same objects as $\ca{C}$ and $\ca{C}^{T}(A,B)\defl \ca{C}(A,TB)$. The composition is defined using the monad multiplication. Show that $\ca{C}^{T}$ is equivalent to the full subcategory of $T\mbox{-}\Alg$ consisting of the free algebras. 
\end{question}

\begin{question}
 Let $\ca{K}$ be a 2-category with EM-objects and let $C \in \ca{K}$. Write $\mathrm{Mnd}(C)$ for the category of monads on $C$. Call a 1-cell $g \colon A \rightarrow C$ \emph{tractable} if the right Kan extension of $g$ along itself exists. Write $\ca{K}_0^{\prime} \slash C$ for the full subcategory of the slice (1-)category consisting of tractable 1-cells. Show that the functor
 \[
(-)\mbox{-}\Alg \colon \mathrm{Mnd}(C)^{\op} \rightarrow \ca{K}_0^{\prime} \slash C
 \]
which sends $t$ to its EM-object is right adjoint to the functor which sends $g$ to the density monad $\Ran_g g$.
\end{question}


\begin{question}
 Let $\ca{C}$ be an unenriched cocomplete category. Let $\ca{G}$ be a full subcategory consisting of $\kappa$-presentable objects and suppose that $\ca{G}$ is a \emph{strong} generator: the functor 
 \[
 \widetilde{K}=\mathrm{Hom}_{\ca{G}}(K,-) \colon \ca{C} \rightarrow [\ca{G}^{\op},\Set]
 \]
 is conservative. The goal of this exercise is to show that $\ca{C}$ is locally $\kappa$-presentable, that is, there automatically exists a \emph{dense} generator of $\kappa$-small  objects. More precisely, let $\ca{A}$ be the closure of $\ca{G}$ under $\kappa$-small colimits. Show that $\ca{A}$ is dense using the following steps.
 
 \begin{enumerate}
 \item[(a)] Show that the category $\ca{A} \slash C$ is $\kappa$-filtered and that the canonical diagram $\ca{A} \slash C \rightarrow \ca{C}$ is sent to a colimit diagram by the functor $\widetilde{K}$.
 \item[(b)] Show the following fact about conservative functors $F \colon \ca{A} \rightarrow \ca{B}$: if $\ca{A}$ has colimits of a given shape, $F$ preserves colimits of that shape, and a specific cocone is sent to a colimit cocone by $F$, then the cocone in question is already a colimit cocone. In other words, a conservative functor \emph{detects} all the colimits that exist in the domain and that it preserves.
 \item[(c)] Conclude that $\ca{A}$ is dense and thus that $\ca{C}$ is locally $\kappa$-presentable.
 \end{enumerate}
 \end{question}

\begin{question}[subtitle=(bonus)]
 Let $\ca{V}$ be a cosmos. Given two \emph{small} $\ca{V}$-categories, show that there exists a $\ca{V}$-category $[\ca{A},\ca{B}]$ whose underlying category is $\Vcat(\ca{A},\ca{B})$. The hom-object can be defined using the usual equalizer in $\ca{V}$ which would give natural transformations for $\ca{V}=\Set$. Show that this  defines an internal hom-object in the monoidal category $\Vcat$, that is, $- \otimes \ca{A} \dashv [\ca{A},-]$.
\end{question}



\end{document}
