\documentclass[a4paper,11pt,twoside, openany]{book}
\usepackage[utf8]{inputenc}
\usepackage{graphicx}
\usepackage{mathrsfs}
\usepackage{amsbsy}
\usepackage{fontenc}
\usepackage{amsfonts}
\usepackage{amsmath} 	
\usepackage{mdframed}
\usepackage{amsthm}  	
\usepackage{amssymb}
\usepackage{amscd}
\usepackage{faktor}
\usepackage{mathtools}
\usepackage{epigraph}
\usepackage{tikz}
\usetikzlibrary{matrix,arrows,decorations.pathmorphing}
\usepackage{tikz-cd}
\usepackage[titletoc]{appendix}
\usepackage{centernot}
\usepackage{bbding}
\usepackage{indentfirst}
\usepackage{hyperref}
\usepackage{xspace}
\hypersetup{colorlinks=false, pdfborder={0 0 0}}                                 
\usepackage{enumitem}
\setdescription{font=\normalfont}
%\usepackage[style=alphabetic, backend=bibtex]{biblatex}
\usepackage[normalem]{ulem}
\usepackage{contour}
\makeatletter        
\newcommand*{\doublerightarrow}[2]{\mathrel{
		\settowidth{\@tempdima}{$\scriptstyle#1$}
		\settowidth{\@tempdimb}{$\scriptstyle#2$}
		\ifdim\@tempdimb>\@tempdima \@tempdima=\@tempdimb\fi
		\mathop{\vcenter{
				\offinterlineskip\ialign{\hbox to\dimexpr\@tempdima+1em{##}\cr
					\rightarrowfill\cr\noalign{\kern.5ex}
					\rightarrowfill\cr}}}\limits^{\!#1}_{\!#2}}}
\newcommand*{\triplerightarrow}[1]{\mathrel{
		\settowidth{\@tempdima}{$\scriptstyle#1$}
		\mathop{\vcenter{
				\offinterlineskip\ialign{\hbox to\dimexpr\@tempdima+1em{##}\cr
					\rightarrowfill\cr\noalign{\kern.5ex}
					\rightarrowfill\cr\noalign{\kern.5ex}
					\rightarrowfill\cr}}}\limits^{\!#1}}}
\makeatother

\def\cleardoublepage{\clearpage\if@twoside \ifodd\c@page\else  		
	\hbox{}                                                        					
	\vspace*{\fill}                                                					
	\begin{center}                                                 					
		\*                                                             					
	\end{center}                                                   					
	\vspace{\fill}                                                 					 
	\thispagestyle{empty}                                          				
	\newpage                                                       					
	\if@twocolumn\hbox{}\newpage\fi\fi\fi}                        

\makeatletter
\newcommand{\colim@}[2]{%
	\vtop{\m@th\ialign{##\cr
			\hfil$#1\operator@font colim$\hfil\cr
			\noalign{\nointerlineskip\kern-\ex@}\cr}}%
}
\newcommand{\colim}{%
	\mathop{\mathpalette\colim@{\rightarrowfill@\scriptscriptstyle}}\nmlimits@
}


\newcommand{\catname}[1]{{\normalfont\textbf{#1}}}

\DeclareMathOperator{\Aff}{\mathbf{Aff}}
\DeclareMathOperator{\Alg}{\mathbf{Alg}}
\DeclareMathOperator{\Coh}{\mathbf{Coh}}
\DeclareMathOperator{\QCoh}{\mathbf{QCoh}}
\DeclareMathOperator{\VB}{\mathbf{VB}}

\newcommand{\Set}{\catname{Set}}
\newcommand{\CHTop}{\catname{CHTop}}
\newcommand{\Ab}{\catname{Ab}}
\newcommand{\RMod}{\catname{Mod}_R}
\newcommand{\sSet}{\catname{sSet}}
\newcommand{\Rel}{\catname{Rel}}
\newcommand{\Cat}{\catname{Cat}}
\newcommand{\CAT}{\catname{CAT}}	
\newcommand{\qCat}{\catname{qCat}}	
\makeatother                                                   					

\makeatletter
\renewcommand\part{%
	\if@openright
	\cleardoublepage
	\else
	\clearpage
	\fi
	\thispagestyle{empty}%  				 
	\if@twocolumn
	\onecolumn
	\@tempswatrue
	\else
	\@tempswafalse
	\fi
	\null\vfil
	\secdef\@part\@spart}
\makeatother
\usepackage{bbm}
\usepackage{fancyhdr}                                   	

\renewcommand{\sectionmark}[1]{\markright{#1}}         	         
\pagestyle{fancy}                                       				
\fancyhf{}                                              				
\fancyhead[LE,RO]{\thepage}                           		           
\fancyhead[LO]{\scshape\nouppercase{\rightmark}}       	          
\fancyhead[RE]{\scshape\nouppercase{\leftmark}}      	           
\renewcommand{\headrulewidth}{0pt}                   \usepackage{amsmath,calligra,mathrsfs}
\DeclareMathOperator{\innerhom}{\mathscr{H}\text{\kern -3pt {\calligra\Large om}}\,}
\DeclareMathOperator{\Hom}{\text{Hom}}
\DeclareMathOperator{\End}{\text{End}}
\DeclareMathOperator{\op}{\text{op}}
\DeclareMathOperator{\co}{\text{co}}
\DeclareMathOperator{\coop}{\text{coop}}
\DeclareMathOperator{\Fib}{\text{Fib}}
\DeclareMathOperator{\Cof}{\text{Cof}}
\DeclareMathOperator{\V}{\mathcal{V}}
\DeclareMathOperator{\A}{\mathbf{A}}
\DeclareMathOperator{\C}{\mathbf{C}}
\DeclareMathOperator{\D}{\mathbf{D}}
\DeclareMathOperator{\N}{\mathbb{N}}
\DeclareMathOperator{\Z}{\mathbb{Z}}
\DeclareMathOperator{\id}{id}
\DeclareMathOperator{\dom}{dom}
\DeclareMathOperator{\cod}{cod}
\DeclareMathOperator{\Ob}{Ob}
\DeclareMathOperator{\Ar}{Ar}

\tikzset{shorten <>/.style={shorten >=#1,shorten <=#1}}
\newcommand{\pullbackcorner}[1][ul]{\save*!/#1+1.5pc/#1(1,-1)@^{|-}\restore}
\newcommand{\pushoutcorner}[1][ul]{\save*!/#1-1.5pc/#1:(-1,1)@^{|-}\restore}
\usepackage{tikz}
\usetikzlibrary{shapes}
\usepackage{xcolor}
\usepackage{url}
\usepackage{attachfile} 							
\makeatletter                        						
\g@addto@macro{\UrlBreaks}{\UrlOrds} 				
\makeatother                        						 
\usepackage{xcolor}
\usepackage{url}
\usepackage{attachfile} 							


\makeatletter                        						

\g@addto@macro{\UrlBreaks}{\UrlOrds} 				

\makeatother                        						 

\newsavebox\MBox

\newcommand\Cline[2][red]{{\sbox\MBox{$#2$}%
		
		\rlap{\usebox\MBox}\color{#1}\rule[-1.2\dp\MBox]{\wd\MBox}{1pt}}}

\theoremstyle{definition}

\newtheorem{thm}{Theorem}[section] % reset theorem numbering for each chapter

\newcommand{\chaptercontent}{
	\section{Basics}
	\begin{defn}Here is a new definition.\end{defn}
	\begin{thm}Here is a new theorem.\end{thm}
	\begin{exmp}Here is a good example.\end{exmp}
	\subsection{Some tips}
	\begin{defn}Here is a new definition.\end{defn}
	\section{Advanced stuff}
	\begin{defn}Here is a new definition.\end{defn}
	\subsection{Warnings}
	\begin{defn}Here is a new definition.\end{defn}
}
\theoremstyle{definition}
\newtheorem{defn}[thm]{Definition} % definition numbers are dependent on theor$
\newtheorem{exmp}[thm]{Example} % same for example numbers
\newtheorem{prop}[thm]{Proposition}
\newtheorem{lemma}[thm]{Lemma}
\newtheorem{cor}[thm]{Corollary}
\theoremstyle{remark}
\newtheorem{rmk}[thm]{Remark}
\newtheorem{es}[thm]{Example}

\newmdtheoremenv{theo}[thm]{Theorem}

\newmdtheoremenv{teo}{Theorem}

\setlength{\headheight}{15pt} 					  
\usepackage[all]{xy}
\usepackage{extarrows}


\begin{document}
	
	\author{by \\
		Nicola Di Vittorio \\ Matteo Durante}
	\title{\huge Monads and their applications \\
		\vspace*{5mm}
		\large Dr.\ Daniel Schäppi's course lecture notes} 
	\date{}
	
	\frontmatter
	\maketitle
	\tableofcontents
	
	\chapter{Introduction}
	
	\mainmatter
	
	\chapter{Categorical preliminaries}
	\begin{defn}[Categories]
		A category $\C$ consists of:
		\begin{enumerate}
			\item a collection of objects $\Ob(\C)$;
			\item a collection of arrows $\Ar(\C)$;
			\item two maps $\dom,\cod\colon\Ar(\C)\rightarrow\Ob(\C)$;
			\item a map $\id_{-}\colon\Ob(\C)\rightarrow\Ar(\C)$ with $\dom(\id_{c})=c=\cod(\id_{c})$;
			\item for every $f,g\in\Ar(\C)$ such that $\cod(f)=\dom(g)$ a unique composite morphism $gf$ such that $\cod(gf)=\cod(g)$, $\dom(gf)=f$.
		\end{enumerate}
	This data has to satisfy the following axioms
		\begin{enumerate}
			\item given $f\in\Ar(\C)$, $c=\dom(f)$ and $c'=\cod(f)$, $\id_{c'}f=f=\id_{c}$, that is the composition is unital;
			\item given a composable triple $f,g,h\in\Ar(\C)$, $h(gf)=(hg)f$, that is the composition is associative.
		\end{enumerate}
		An arrow $f$ such that $c=\dom(f)$ and $c'=\cod(f)$ is denoted $f\colon c\rightarrow c'$.
	\end{defn}
	
	\begin{defn}[Functors]
		
	\end{defn}
	
	\begin{defn}[Full functors, faithful functor]
		
	\end{defn}
	
	\begin{defn}[Natural transformations]
		
	\end{defn}
	
	\begin{defn}[Equivalent functors]
	\end{defn}
	
	\begin{defn}[Representable Functors]
		
	\end{defn}
	
	\begin{defn}[Whiskering]
		
	\end{defn}
	
	\begin{defn}[Horizontal and vertical composition of nat.transf.]
		
	\end{defn}
	
	\begin{defn}[adjunctions]
		
	\end{defn}
	
	\begin{lemma}[Yoneda]
		
	\end{lemma}
	\begin{proof}
		
	\end{proof}
	
	\chapter{Monads and algebras}
	
	\section{Introduction}
		
	Throughout mathematics we encounter structures defined by some action morphisms. Here we give some examples.
	
	\begin{exmp}
		Given a group $G$, we may consider a $G$-set $X$ described by an action map $G\times X\rightarrow X$.
	\end{exmp}
	\begin{exmp}
		Given an abelian group $M$ and a ring $R$, we can get an $R$-module $M$ by fixing a group homomorphism $R\otimes_{\Z} M\rightarrow M$.
	\end{exmp}
	\begin{exmp}
		Given a monoid $M$ in $\Set$, we get a map $\Pi_{k=1}^n M\rightarrow M$, $(m_1,\ldots,m_n)\mapsto ((\ldots ((m_1m_2)m_3)\ldots )m_{n-1}) m_n$. This induces an action map from $W(M)=\amalg_{n\in\N}\Pi_{k=1}^n M$, the set of words on $M$, to $M$.
	\end{exmp}
	\begin{exmp}\label{ultrafilters}
		Given a set $X$, let $\mathcal{U}X$ be the set of ultrafilters on it. Any compact T2 topology on $X$ allows us to see each ultrafilter as a system of neighborhoods of a unique point in $X$, hence it gives us a unique map $\mathcal{U}X\rightarrow X$ sending each ultrafilter to the respective point.
	\end{exmp}
	\begin{exmp}
		Given a directed graph $D=(V,E,\hspace{-1.5mm}\begin{tikzcd}
		E\hspace{-1.5mm}\arrow[r, "t" description,  shift right=.5ex, shorten <= .3em, shorten >= .3em]  \arrow[r, "s" description, shift left=1ex, shorten <= .3em, shorten >= .3em] & \hspace{-1.5mm}V
		\end{tikzcd}\hspace{-1.5mm})$, we can create its free category $FD$, where the objects are the vertices and $FD(v,w)=\{\text{finite paths } v\rightarrow\ldots\rightarrow w\}$. We set $\id_v$ to be the path of length 0, while composition is just the concatenation of paths.
		
		In particular, if $D$ is the directed graph with $V=\{0,\ldots,n\}$ and an edge $j\rightarrow k$ if and only if $k=j+1$, we have $FD\cong [n]$.
		
		If $D=\{*\}$ and $E=\{*\rightarrow *\}$, then $FD(*,*)\cong\N$.
		
		Given a small category $\C$, we may consider the underlying graph $U\C=D$ with $V=\Ob(\C)$, $E=\Ar(\C)$, $s=\dom$ and $t=\cod$. We get then an action map $UFU\C\rightarrow U\C$ sending a finite path to its composite. This map is a morphism of directed graphs.
	\end{exmp}

	Notice that we always have a category $\C$ and some functor $T\colon\C\rightarrow\C$ with an action map $T\C\rightarrow\C$. How can we see all of these examples as specific instances of a general phenomenon?
	
	\begin{defn}
		A monad on a category $\C$ is a triple $(T,\mu,\eta)$ where $T\colon\C\rightarrow\C$ is a functor, while $\mu\colon T^2\Rightarrow T$ and $\eta\colon\id_{\C}\Rightarrow T$ are natural transformations such that the following diagrams commute:
		
		\[
			\begin{tikzcd}
				T^3\ar["\mu T", Rightarrow]{d}\ar["T\mu", Rightarrow]{r}
				& T^2\ar["\mu", Rightarrow]{d} \\
				T^2\ar["\mu", Rightarrow]{r}
				& T
			\end{tikzcd}
			\quad\quad
			\begin{tikzcd}
				T\ar["\eta T", Rightarrow]{r}\ar["\id_{T}", Rightarrow, swap]{dr}
				& T^2\ar["\mu", Rightarrow]{d}\ar["T\eta", Leftarrow]{r}
				& T\ar["\id_{T}", Rightarrow]{dl} \\
				& T
			\end{tikzcd}
		\]
		
		$\mu$ is called the multiplicative map, while $\eta$ is the unit of $T$.
		
		The commutativity of the first diagram is equivalent to stating that the following two diagrams are equal:
		
		\begin{minipage}{0.3\linewidth}
			\begin{tikzcd}[row sep=1cm, column sep=1cm]
				&\C\ar[d, Rightarrow, shorten <= 1em, shorten >= 1em, "\mu"]\ar[r, "T"]\ar[drr, bend right=26, "T"description]
				&\C\ar[dr, "T"]\ar[d, Rightarrow, yshift=1ex, shorten <= 1em, shorten >= 1em, "\mu"]\\
				\C
				\ar[rrr, "T"'] 
				\ar[ur, , "T"]
				&\phantom{.} &\phantom{.}&\C
			\end{tikzcd}
		\end{minipage}
		\hspace{1.5cm}
		=
		\begin{minipage}{0.3\linewidth}
			\begin{tikzcd}[row sep=1cm, column sep=1cm]
				&\C\ar[d, Rightarrow, yshift=1ex, shorten <= 1em, shorten >= 1em, "\mu"]\ar[r, "T"]
				&\C\ar[d, Rightarrow, shorten <= 1em, shorten >= 1em, "\mu"]\ar[dr, "T"]\\
				\C\ar[urr, bend right=26, "T"'description]
				\ar[rrr, "T"'] 
				\ar[ur, , "T"]
				&\phantom{.} &\phantom{.}&\C
			\end{tikzcd}
		\end{minipage}
	\end{defn}

On the other hand, the second diagram can be rephrased as follows:
\begin{center}
	
	\begin{minipage}{0.3\linewidth}
		\begin{tikzcd}[row sep=1cm, column sep=1cm]
		& \C \arrow[d, Rightarrow, shift left=.5ex, shorten <= 1em, shorten >= 1em, "\mu"]\arrow[d, Rightarrow, shift right=5.8ex, shorten <= 1em, shorten >= 1em, "\eta"]\arrow[rd, "T"] &   \\
		\C \arrow[rr, "T"', ""{name=A}] \arrow[ru, bend right, "T"'description, ""{name=T}] \arrow[ru, bend left, "\id_{\C}", ""{name=U}] &    \phantom{.}          & \C
		%\arrow[Rightarrow, from=U, to=D]
		\end{tikzcd}
		
	\end{minipage}
	=
	\hspace{.2cm}
	\begin{minipage}{0.3\linewidth}
		\begin{tikzcd}[row sep=1cm, column sep=1cm]
		\C \arrow[d, bend right, "T"'] \arrow[d, bend left, "T"] \\
		\C                                           
		\end{tikzcd}
	\end{minipage}
	\hspace{-3.5cm}
	=
	\hspace{5mm}
	=
	\begin{minipage}{0.3\linewidth}
		\begin{tikzcd}[row sep=1cm, column sep=1cm]
		& \C \arrow[d, Rightarrow, shift right=1.2ex, shorten <= 1em, shorten >= 1em, "\mu"]\arrow[d, Rightarrow, shift left=5.8ex, shorten <= 1em, shorten >= 1em, "\eta"] \arrow[rd, bend right, "T"'description, ""{name=T}] \arrow[rd, bend left, "\id_{\C}", ""{name=U}] &   \\
		\C \arrow[rr, "T"', ""{name=A}] \arrow[ru, "T", ""{name=B}] &\phantom{.}   & \C		\end{tikzcd}
	\end{minipage}
\end{center}

	A monad naturally defines other algebraic structures, which we now introduce.
	
	\begin{defn}
		Given a monad $(T,\mu,\eta)$, a $T$-algebra or $T$-module is a pair $(a,\alpha)$, where $a\in\Ob(\C)$ and $\alpha\colon Ta\rightarrow a$ is such that the following diagrams commute:
		
		\[	
			\begin{tikzcd}
				T^2 a\ar["\mu_{a}"]{d}\ar["T\alpha"]{r}
				& Ta\ar["\alpha"]{d} \\
				Ta\ar["\alpha"]{r}
				& a
			\end{tikzcd}
			\quad\quad
			\begin{tikzcd}
				a\ar["\id_{a}", swap]{dr}\ar["\eta_{a}"]{r}
				& Ta\ar["\alpha"]{d} \\
				& a
			\end{tikzcd}
		\]
	\end{defn}

	\begin{defn}
		A morphism of $T$-algebras $(a,\alpha)\rightarrow (b,\beta)$ is a morphism $f\colon a\rightarrow b$ such that the following diagram commutes:
		\[
			\begin{tikzcd}
				Ta\arrow["\alpha"]{d}\arrow["Tf"]{r}
				& Tb\arrow["\beta"]{d} \\
				a\arrow["f"]{r}
				& b
			\end{tikzcd}
		\]
	\end{defn}

		
	$T$-algebras form a category $T\mbox{-}\Alg$, which has a natural forgetful functor $U^T\colon T\mbox{-}\Alg\rightarrow\C$.
	
	We now show how to recover the examples previously given with this language.
	
	\begin{exmp}
		\begin{align*}
			T=G\times - \colon&\ \Set \rightarrow\Set \\
			\mu_A\colon&\ G\times (G\times A) \rightarrow G\times A \\
			&\ (g,(h,a)) \mapsto (gh,a) \\
			\eta_A\colon&\ A \rightarrow G\times A \\
			&\ a \mapsto (e,a)
		\end{align*}
		is a monad and $(A,\alpha)$ is a $T$-algebra if and only if $A$ is a $G$-set. It follows that $T\mbox{-}\Alg\cong G\mbox{-}\Set$.
	\end{exmp}

	\begin{exmp}
		Given a ring $R$, $T=R\otimes_{\Z}-\colon \Ab\rightarrow \Ab$ is a monad when considered with the following natural transformations:
		
		\begin{align*}
			\mu_{-}\colon &\ R\otimes_{\Z}(R\otimes_{\Z}-)\cong (R\otimes_{\Z})\otimes_{\Z}-\Rightarrow R\otimes_{\Z}- \\
			\eta_-\colon &\ -\cong\Z\otimes_{\Z}-\Rightarrow R\otimes_{\Z}-
		\end{align*}
	
		We have that $(R\otimes_{\Z}-)\mbox{-}\Alg\cong \RMod$.
	\end{exmp}

	\begin{exmp}
		Consider $W\colon\Set\rightarrow\Set$ given by $WX=\amalg_{n\in\N}\Pi_{k=1}^n X$. Multiplication $\mu_X\colon WWX\rightarrow WX$ is given by concatenation of words, while the unit $\eta_X\colon X\rightarrow WX$ is just $x\mapsto (x)$. With this, $W\mbox{-}\Alg\cong \text{Mon}(\Set)$.
	\end{exmp}

	\begin{exmp}
		The functor $\mathcal{U}$ defined in Example \ref{ultrafilters}, equipped with suitable natural transformations, is a monad on $\Set$ and $\mathcal{U}\mbox{-}\Alg\cong \CHTop$, the category of compact T2 spaces.
	\end{exmp}

	\begin{exmp}
		The free-forgetful adjunction $F\dashv U$ between categories and directed graphs induces a monad on the latter, with $UF\mbox{-}\Alg\cong\Cat$.
	\end{exmp}

	\section{Monadic functors}
	
	Now that we have introduced these structures, our aim is to characterize \emph{monadic functors}, that is functors $U\colon\A\rightarrow\C$ which are equivalent to $U^T\colon T\mbox{-}\Alg\rightarrow\C$ for some monad $(T,\mu,\eta)$ on $\C$.

	First of all, notice that $U^T$ is faithful by construction, hence $U$ must be faithful, but more is true.
	
	\begin{lemma}
		The functor $U^T$ is conservative, that is if $U^Tf$ is an isomorphism then $f$ is an isomorphism of $T$-algebras.
	\end{lemma}
	\begin{proof}
		Suppose that $g$ is the inverse of $f\colon a\rightarrow b$ and $f$ is a morphism $(a,\alpha)\rightarrow (b,\beta)$. We only need to prove that the square on the left commutes, that is $g\beta=\alpha Tg$:
		\[
			\begin{tikzcd}
				Tb\arrow["\beta"]{d}\arrow["Tg"]{r}
				& Ta\arrow["\alpha"]{d}\arrow["Tf"]{r}
				& Tb\arrow["\beta"]{d} \\
				b\arrow["g"]{r}
				& a\arrow["f"]{r}
				& b
			\end{tikzcd}
		\]
		
	We see that $fg\beta=\beta$ and $f\alpha Tg=\beta Tf Tg=\beta T(fg)=\beta T\id_b=\beta$, hence the thesis.
	\end{proof}
    \begin{rmk}
     Notice that the forgetful functor $U\colon\mathbf{Top}\to\Set$ can't be monadic since it does not reflect isomorphisms. However, if we restrict it to the full subcategory of $\mathbf{Top}$ spanned by compact Hausdorff spaces we indeed obtain a monadic functor.
    \end{rmk}
	\begin{prop}
		The functor $U^T\colon T\mbox{-}\Alg\rightarrow\C$ has a left adjoint $F^T\colon\C\rightarrow T\mbox{-}\Alg$ such that $F^Tc=(Tc,\mu_{c})$, $F^Tf\colon(Tc,\mu_{c})\xrightarrow{Tf} (Td,\mu_{d})$ and $U^TF^T=T$. Furthermore, the unit of this adjunction is given by $\gamma_c=\eta_c\colon c\to U^TF^Tc=Tc$ and the counit has components $\epsilon_{(a,\alpha)}=\alpha\colon(Ta,\mu_a)\to(a,\alpha)$.
	\end{prop}
\begin{proof}
	\begin{enumerate}[label=(\roman*)]
		\item To show that $(Tc, \mu_c)$ is a $T$-algebra we need the following diagrams to be commutative.
		\[
		\begin{tikzcd}
		T^3c\ar["\mu_{Tc}"', rightarrow]{d}\ar["T\mu_c", rightarrow]{r}
		& T^2c\ar["\mu_c", rightarrow]{d} \\
		T^2c\ar["\mu_c"', rightarrow]{r}
		& Tc
		\end{tikzcd}
		\quad\quad
		\begin{tikzcd}
		Tc\ar["\eta_{Tc}",rightarrow]{r}\ar[equal]{dr}
		& T^2c\ar["\mu_c", rightarrow]{d}
		\\
		& Tc
		\end{tikzcd}
		\]
		These are exactly the associativity and one of the unit laws for $(T, \mu, \eta)$.
	\item For every $f\colon c\to c'$, $Tf$ is a morphism of algebras $(Tc,\mu_c)\to(Tc', \mu_{c'})$. The diagram 
		\[
	\begin{tikzcd}
	T^2c\ar["\mu_{c}"', rightarrow]{d}\ar["T^2f", rightarrow]{r}
	& T^2c'\ar["\mu_{c'}", rightarrow]{d} \\
	Tc\ar["Tf"', rightarrow]{r}
	& Tc'
	\end{tikzcd}
		\]
	is commutative because of the naturality of $\mu$. Hence $F^T$ is defined on morphisms. It is a functor by the functoriality of $T$.
	\item The unit is natural by assumption. We claim that $\epsilon_{(a,\alpha)}=\alpha$ is a morphism of algebras $$F^TU^T(a,\alpha)=F^Ta=(Ta,\mu_a) \to \id_{T\mbox{-}\Alg}(a,\alpha)=(a,\alpha)$$ and $\epsilon$ is a natural transformation $F^TU^T\Rightarrow\id_{T\mbox{-}\Alg}$. Let's check it. We know that $\alpha$ is a morphism of algebras if and only if 
	\[	
	\begin{tikzcd}
	T^2 a\ar["\mu_{a}"']{d}\ar["T\alpha"]{r}
	& Ta\ar["\alpha"]{d} \\
	Ta\ar["\alpha"']{r}
	& a
	\end{tikzcd}
	\]
	is commutative. But this is one of the two $T$-algebra axioms! Moreover, to prove that $\epsilon$ is natural, we need to show that
		\[	
	\begin{tikzcd}
	(Ta,\mu_a) \ar["Tf"']{d}\ar["\alpha=\epsilon_{(a,\alpha)}"]{r}
	& (a,\alpha)\ar["f"]{d} \\
	(Tb,\mu_b)\ar["\beta=\epsilon_{(b,\beta)}"']{r}
	& (b,\beta)
	\end{tikzcd}
	\]
	is commutative, but this is the axiom for $f$ to be a morphism of $T$-algebras! 
	\item It remains to check the two triangular identities $\epsilon F^T \circ F^T\eta=id_{F^T}$ and $U^T\epsilon\circ\eta U^T=id_{U^T}$. These are to be checked on the components at $c$ and $(a, \alpha)$, respectively.
	\[
		\begin{tikzcd}
	(Tc,\mu_c)\ar["T\eta_{c}",rightarrow]{r}\ar[equal]{dr}
	& (T^2c,\mu_{Tc})\ar["\mu_{Tc}", rightarrow]{d}
	\\
	& (Tc,\mu_c)
	\end{tikzcd}	\quad\quad
	\begin{tikzcd}
	a\ar["\eta_a",rightarrow]{r}\ar[equal]{dr}
	& Ta\ar["\alpha", rightarrow]{d}
	\\
	& a
	\end{tikzcd}
	\]
	The commutativity of these two diagrams is ensured by the second unit law for a monad and the unit law for the $T$-algebra $(a,\alpha)$, respectively.
	\end{enumerate}
\end{proof}
\begin{defn}
Algebras of the form $(Tc, \mu_c)$ are called free $T$-algebras.	
\end{defn}
Thanks to the proposition above we can prove that, given a monad $T$ we can always find an adjunction that generates it. Actually, the converse holds too.
\begin{prop}
If $U\colon\D\to\C$ has a left adjoint $F$ with unit $\eta$ and counit $\epsilon$, then $(UF,U\epsilon F,\eta)$ is a monad on $\C$. Also, if $(T,\mu,\eta)$ is a monad on $\C$, then $(U^TF^T,U^T\epsilon F^T,\eta)=(T,\mu,\eta)$. 
\end{prop}
\begin{proof}
Let us check the axioms. First of all, the associativity holds due to the following equations.
\flushleft
	\begin{tikzcd}[row sep=.55cm, column sep=.55cm]
		&&\C\arrow[rd, "F"]\ar[d, Rightarrow, shorten <= .5em, shorten >= .5em, "\epsilon"']&&&&&&\C\arrow[rd, "F"]\ar[d, Rightarrow, shorten <= .5em, shorten >= .5em, "\epsilon"'] &&\\
		\C\arrow[r, "F"]&\D\arrow[ru, "U"] \arrow[rr, equal] &\phantom{.}& \D \arrow[r, "U"] & \C \arrow[r, "F"] & \D \arrow[r, "U"] & \C  \arrow[r, "F"] &\D\arrow[ru, "U"] \arrow[rr, equal] &\phantom{.}& \D \arrow[r, "U"] & \C
		\end{tikzcd}
			\begin{center}
				=				
			\end{center}
			\begin{center}
			\begin{tikzcd}
		&&\C\arrow[rd, "F"]\ar[d, Rightarrow, shorten <= .5em, shorten >= .5em, "\epsilon"']&&\C\arrow[rd, "F"]\ar[d, Rightarrow, shorten <= .5em, shorten >= .5em, "\epsilon"']\\
		\C\arrow[r, "F"]&\D\arrow[ru, "U"] \arrow[rr, equal] &\phantom{.}& \D\arrow[ru, "U"] \arrow[rr, equal] &\phantom{.}& \D\arrow[r, "U"]&\C
		\end{tikzcd}	
	\end{center}
\begin{center}
	=
		\begin{tikzcd}[row sep=.55cm, column sep=.55cm]
		&            &            &                                    &\C  \arrow[rd, "F"]\ar[d, Rightarrow, shorten <= .5em, shorten >= .5em, "\epsilon"'] &            &            &                                    &  \C\arrow[rd, "F"]\ar[d, Rightarrow, shorten <= .5em, shorten >= .5em, "\epsilon"'] &            &  \\
		\C\arrow[r, "F"] & \D \arrow[r, "U"] & \C  \arrow[r, "F"] &  \D\arrow[ru, "U"] \arrow[rr, equal] &  \phantom{.}           &\D  \arrow[r, "U"] &\C  \arrow[r, "F"] & \D \arrow[ru, "U"] \arrow[rr, equal] &\phantom{.}             &\D  \arrow[r, "U"] & \C
		\end{tikzcd}
\end{center}
Unit laws:
\[
\begin{tikzcd}
&         \arrow[d, Rightarrow, yshift=1ex, shorten <= .5em, shorten >= .01em, "\eta"']    &\C \arrow[d, Rightarrow, shorten <= .5em, shorten >= .5em, "\epsilon"']  \arrow[rd, "F"] &            &  \\
\C\arrow[r, "F"] \arrow[rru, bend left, equal] &\D  \arrow[ru, "U"] \arrow[rr, equal] &  \phantom{.}           &\D  \arrow[r, "U"] & \C
\end{tikzcd}
\]
is equal to $1_{UF}$, since $\epsilon F\circ F\eta=1_F$ by one of the triangular identities of the adjunction $F\dashv U$. Furthermore,
\[
\begin{tikzcd}
&            &\C \arrow[d, Rightarrow, shorten <= .5em, shorten >= .5em, "\epsilon"']  \arrow[rd, "F"]\arrow[rrd, bend left, equal] &  \arrow[d, Rightarrow, yshift=1ex, shorten <= .5em, shorten >= .01em, "\eta"']           &  \\
\C\arrow[r, "F"]  &\D  \arrow[ru, "U"] \arrow[rr, equal] &  \phantom{.}           &\D  \arrow[r, "U"] & \C
\end{tikzcd}
\] 
is equal to $1_{UF}$. This follows from the explicit description of the unit and the counit of the adjunction $F^T\dashv U^T$, in fact 
$$U^T\epsilon F^Tc=U^T\epsilon_{(Tc, \eta_c)}=\mu_c$$ \qedhere
\end{proof}
\begin{exmp}[Interesting adjunction, boring monad] Let us consider the adjunction $U\colon\mathbf{Top}\rightleftarrows\Set\colon\mathsf{Disc}\eqqcolon F$, whose left adjoint assigns to every set $X$ the discrete topological space $FX=(X, 2^X)$.
It's immediate to see that $UFX=X$, hence $UF=\id_\Set$. How many natural transformations $\id_\Set=UF\xRightarrow{\alpha} UF=\id_\Set$ are there?
We know that $\id_\Set\cong\hom(*,-)$, so $\text{Nat}(\id_\Set, \id_\Set)\cong\text{Nat}(\hom(*,-),\hom(*,-))\cong\hom(*,*)=\{\id_*\}$ by Yoneda, hence $\alpha=\id$. Therefore $(UF,U\epsilon F, \eta)=(\id_\Set, \id, \id)$
\end{exmp}
\begin{exmp}
If $S$ is a set, $\Set(S,-)\colon\Set\to\Set$ is right adjoint to $S\times-\colon\Set\to\Set$, so we get a monad $X\mapsto\Set(S,S\times X)$. This is called \emph{the state monad} and is important in Computer Science.
\end{exmp}
There is always a comparison morphism $\D\xrightarrow{\overline{U}}UF\mbox{-}\Alg$	s.t. 
\[
\begin{tikzcd}
\D\ar["\overline{U}"]{rr}\ar["U"']{dr}
&& UF\mbox{-}\Alg\ar["U^{UF}", rightarrow]{dl}
\\
& \C
\end{tikzcd}
\]
commutes. We set $\overline{U}f=(Ud,UFUd\xrightarrow{U\epsilon_d}Ud)=(Ud, U\epsilon_d)$. More generally, for a given functor $\D\xrightarrow{G}\C$ we can ask what do we need to get a lift $\overline{G}\colon\D\to T\mbox{-}\Alg$.
\begin{defn}
A $T$-action on a functor $\D\xrightarrow{G}\C$ is a natural transformation $TG\xRightarrow{\gamma}G$ s.t. 
\[
\begin{tikzcd}
T^2G\ar["\mu G"', Rightarrow]{d}\ar["T\gamma", Rightarrow]{r}
& TG\ar["\gamma", Rightarrow]{d} \\
TG\ar["\gamma"', Rightarrow]{r}
& G
\end{tikzcd}
\quad\quad
\begin{tikzcd}
G\ar["\eta G", Rightarrow]{r}\ar[equal]{dr}
& TG\ar["\gamma", Rightarrow]{d}\\
& G
\end{tikzcd}
\]
commute.
\end{defn}
\begin{defn}
A morphism of $T$-actions is a natural transformation $\varphi\colon G\Rightarrow K$ s.t. 
\[
\begin{tikzcd}
TG\ar["T\varphi", Rightarrow]{r}\ar["\gamma"', Rightarrow]{d}
& TK\ar["\alpha", Rightarrow]{d}\\
G\ar[r, Rightarrow, "\varphi"']& K 
\end{tikzcd}
\]
commutes.
\end{defn}
Up to size, we get a category $T$-$\mathsf{Act}(\D)$, defined as follows.

\begin{defn}
Given a monad $(T,\mu, \eta)$
\end{defn}
	\chapter{Beck’s monadicity theorem}
	\begin{thm}[Beck]
	
	\end{thm}
	\chapter{Monads in 2-category theory}
	\chapter{Monads in $\infty$-category theory}
	
	
	
	
	\backmatter
	% bibliography, glossary and index would go here.
	
\end{document}